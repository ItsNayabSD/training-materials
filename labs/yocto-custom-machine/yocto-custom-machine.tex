\subchapter{Lab6: Create a custom machine configuration}{Let Poky know about your
  hardware!}

During this lab, you will:
\begin{itemize}
  \item Create a custom machine configuration
  \item Understand how the target architecture is dynamically chosen
\end{itemize}

\section{Create a custom machine}

The machine file configures various hardware related settings. That's
what we did in lab1, when we chose the \ifdefstring{\labboard}{stm32mp1}
{\code{stm32mp1}} {\ifdefstring{\labboard}{beagleplay}{\code{beagleplay}}{\code{beaglebone}}} one.
While it is not necessary to make our custom machine image here, we'll create a
new one to demonstrate the process.

Add a new \code{bootlinlabs} machine to the previously created layer, which
will make the
\ifdefstring{\labboard}{stm32mp1}{Discovery}{{\ifdefstring{\labboard}{beagleplay}{BeaglePlay}{BeagleBone}}} 
properly boot.

\if\defstring{\labboard}{stm32mp1}
This machine describes a board using the \code{cortexa7thf-neon-vfpv4}
tune and is a part of the \code{stm32mp} SoC family. Add the following
lines to your machine configuration file:

\begin{verbatim}
require conf/machine/include/st-machine-common-stm32mp.inc
require conf/machine/include/st-machine-providers-stm32mp.inc

DEFAULTTUNE = "cortexa7thf-neon-vfpv4"
require conf/machine/include/arm/armv7a/tune-cortexa7.inc
\end{verbatim}
\else
  \if\defstring{\labboard}{beagleplay}
This machine describes a board which is a part of the \code{am62xx} SoC family.
This family is based on the arm64-based TI K3 platform.
Add the following lines to your machine configuration file:

\begin{verbatim}
require conf/machine/include/k3.inc
SOC_FAMILY:append = ":am62xx"
\end{verbatim}

The \code{k3.inc} include defines a lot of useful variables, especially
the \yoctovar{DEFAULTTUNE}.
  \else
This machine describes a board using the \code{cortexa8thf-neon} tune
and is a part of the \code{ti33x} SoC family. Add the following lines
to your machine configuration file:

\begin{verbatim}
require conf/machine/include/ti-soc.inc
SOC_FAMILY:append = ":ti33x"

DEFAULTTUNE = "armv7athf-neon"
require conf/machine/include/arm/armv7a/tune-cortexa8.inc
\end{verbatim}
  \fi
\fi
\section{Populate the machine configuration}

This \code{bootlinlabs} machine needs:

\if\defstring{\labboard}{stm32mp1}
\begin{itemize}
  \item To define a few variables to set to get the tooling from ST
    Micro to work properly:
\begin{verbatim}
UBOOT_CONFIG = "trusted_stm32mp15"
STM32MP_DT_FILES_DK = "stm32mp157a-dk1 stm32mp157d-dk1"
\end{verbatim}
  \item To add \code{m4copro} to \yoctovar{MACHINE_FEATURES}
\else
  \if\defstring{\labboard}{beagleplay}
\begin{itemize}
  \item To define a few variables to set to get the tooling from TI
   to work properly:
\begin{verbatim}
TFA_BOARD = "lite"
TFA_K3_SYSTEM_SUSPEND = "1"
OPTEEMACHINE = "k3-am62x"
UBOOT_MACHINE = "am62x_evm_a53_defconfig"
UBOOT_CONFIG_FRAGMENTS = "am625_beagleplay_a53.config"

PREFERRED_PROVIDER_virtual/kernel = "linux-bb.org"
PREFERRED_PROVIDER_virtual/bootloader = "u-boot-bb.org"
PREFERRED_PROVIDER_u-boot = "u-boot-bb.org"

KERNEL_DEVICETREE = "ti/k3-am625-beagleplay.dtb"

# To not get linux-bb.org skipped because of COMPATIBLE_MACHINE
MACHINEOVERRIDES =. "beagle:"
\end{verbatim}
  \item To add an include which is specific to our labs
  and allows to use the extlinux U-Boot bootflow:
\begin{verbatim}
require conf/machine/include/extlinux-bb.inc
\end{verbatim}
  \else
\begin{itemize}
  \item To select \code{linux-ti-staging} as the preferred provider
    for the kernel.
  \item To build \code{am335x-boneblack.dtb} and the
    \code{am335x-boneblack-wireless.dtb} device trees.
  \item To select \code{u-boot-ti-staging} as the preferred provider
    for the bootloader.
  \item To use \code{arm} as the U-Boot architecture.
  \item To use \code{am335x_evm_config} as the U-Boot
    configuration target.
  \item To use \code{0x80008000} as the U-Boot entry point and load
    address.
  \item To use a \code{zImage} kernel image type.
  \item To configure one serial console to \code{115200;ttyS0}
  \item To support some features:
    \begin{itemize}
      \item \code{apm}
      \item \code{usbgadget}
      \item \code{usbhost}
      \item \code{vfat}
      \item \code{ext2}
      \item \code{alsa}
    \end{itemize}
  \fi
\fi
\end{itemize}

\if\defstring{\labboard}{beagleplay}
\section{Populate the k3r5 machine configuration}

The BeaglePlay has a complex booting flow. In particular, it boots on a dedicated Cortex-R5 CPU,
 which has a different architecture from the main Cortex-A53 CPUs that will, in turn, run Linux.
 To support this in Yocto, we have to create a new machine configuration file with the \code{-k3r5} suffix.
\footnote{Behind the scenes, the meta-ti layer uses an advanced Yocto mechanism called \code{multiconfig}.}

This \code{bootlinlabs-k3r5.conf} machine needs the following lines:
\begin{verbatim}
  require conf/machine/include/k3r5.inc

  PREFERRED_PROVIDER_virtual/bootloader = "u-boot-bb.org"
  PREFERRED_PROVIDER_u-boot = "u-boot-bb.org"

  SYSFW_SOC = "am62x"
  SYSFW_CONFIG = "evm"
  SYSFW_SUFFIX = "gp"

  UBOOT_MACHINE = "am62x_evm_r5_defconfig"
  UBOOT_CONFIG_FRAGMENTS = "am625_beagleplay_r5.config"

  # To not get u-boot-bb.org skipped because of COMPATIBLE_MACHINE
  MACHINEOVERRIDES =. "beagle:"
\end{verbatim}
\fi

\section{Build an image with the new machine}

You can now update the \yoctovar{MACHINE} variable value in the local configuration
and start a fresh build.

\section{Check generated files are here and correct}

Once the generated images supporting the new \code{bootlinlabs} machine are
generated, you can check all the needed images were generated
correctly.

Have a look in the output directory, in
\ifdefstring{\labboard}{beagleplay}{\code{$BUILDDIR/deploy-ti/images/bootlinlabs/}}
{\code{$BUILDDIR/tmp/deploy/images/bootlinlabs/}}.

\if\defstring{\labboard}{beaglebone}
Is there anything missing?
\fi

\section{Update the rootfs}

You can now update your root filesystem, to use the newly
generated image supporting our \code{bootlinlabs} machine!

\if\defstring{\labboard}{beaglebone}
\section{Going further}

We chose a quite generic tune (\code{armv7athf-neon}). It's the same
one as meta-ti's definition for the Beaglebone machine. You can see
what Bitbake did in \code{$BUILDDIR/tmp/work}.

Now, we can change the tune to \code{cortexa8thf-neon}. Rebuild the
image, and look at \code{$BUILDDIR/tmp/work}. What happened?
\fi
