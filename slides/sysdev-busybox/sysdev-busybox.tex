\section{BusyBox}

\begin{frame}
  \frametitle{Why BusyBox?}
  \begin{itemize}
  \item A Linux system needs a basic set of programs to work
    \begin{itemize}
    \item An init program
    \item A shell
    \item Various basic utilities for file manipulation and system
      configuration
    \end{itemize}
  \item In normal GNU/Linux systems, these programs are provided by
    different projects
    \begin{itemize}
    \item \code{coreutils}, \code{bash}, \code{grep}, \code{sed},
      \code{tar}, \code{wget}, \code{modutils}, etc. are all different
      projects
    \item A lot of different components to integrate
    \item Components not designed with embedded systems constraints in
      mind: they are not very configurable and have a wide range of
      features
    \end{itemize}
  \item BusyBox is an alternative solution, extremely common on
    embedded systems
  \end{itemize}
\end{frame}

\begin{frame}
  \frametitle{General purpose toolbox: BusyBox}
  \begin{columns}
    \column{0.75\textwidth}
      \url{https://www.busybox.net/}
      \begin{itemize}
      \item Rewrite of many useful UNIX command line utilities
        \begin{itemize}
        \item Created in 1995 to implement a rescue and installer
         system for Debian, fitting in a single floppy disk (1.44 MB)
        \item Integrated into a single project, which makes it easy to
          work with
        \item Great for embedded systems: highly configurable,
          no unnecessary features
        \item Called the {\em Swiss Army Knife of Embedded Linux}
        \end{itemize}
      \item License: GNU GPLv2
      \item Alternative: Toybox, BSD licensed (\url{https://en.wikipedia.org/wiki/Toybox})
      \end{itemize}
    \column{0.25\textwidth}
    \includegraphics[width=\textwidth]{common/busybox.png}
  \end{columns}
\end{frame}

\begin{frame}
  \frametitle{BusyBox in the root filesystem}
  \begin{columns}
    \column{0.65\textwidth}
      \begin{itemize}
      \item All the utilities are compiled into a single executable,
        \code{/bin/busybox}
        \begin{itemize}
        \item Symbolic links to \code{/bin/busybox} are created for each
          application integrated into BusyBox
        \end{itemize}
      \item For a fairly featureful configuration, less than 500 KB
        (statically compiled with uClibc) or less than 1 MB (statically
        compiled with glibc).
      \end{itemize}
    \column{0.35\textwidth}
      \includegraphics[width=\textwidth]{slides/sysdev-busybox/busybox-tree.png}
  \end{columns}
\end{frame}

\begin{frame}[fragile]
  \frametitle{BusyBox - Most commands in one binary}
  \tiny
  % To update this list, compile busybox with "make allyesconfig" on x86
  % Then run "./busybox" and reduce the terminal width until the first
  % line ends with "blkid". Then copy the output here.
  \begin{verbatim}
[, [[, acpid, add-shell, addgroup, adduser, adjtimex, arch, arp, arping, ash, awk, base64, basename, bc, beep, blkdiscard, blkid,
blockdev, bootchartd, brctl, bunzip2, bzcat, bzip2, cal, cat, chat, chattr, chgrp, chmod, chown, chpasswd, chpst, chroot, chrt,
chvt, cksum, clear, cmp, comm, conspy, cp, cpio, crond, crontab, cryptpw, cttyhack, cut, date, dc, dd, deallocvt, delgroup,
deluser, depmod, devmem, df, dhcprelay, diff, dirname, dmesg, dnsd, dnsdomainname, dos2unix, dpkg, dpkg-deb, du, dumpkmap,
dumpleases, echo, ed, egrep, eject, env, envdir, envuidgid, ether-wake, expand, expr, factor, fakeidentd, fallocate, false,
fatattr, fbset, fbsplash, fdflush, fdformat, fdisk, fgconsole, fgrep, find, findfs, flock, fold, free, freeramdisk, fsck,
fsck.minix, fsfreeze, fstrim, fsync, ftpd, ftpget, ftpput, fuser, getopt, getty, grep, groups, gunzip, gzip, halt, hd, hdparm,
head, hexdump, hexedit, hostid, hostname, httpd, hush, hwclock, i2cdetect, i2cdump, i2cget, i2cset, i2ctransfer, id, ifconfig,
ifdown, ifenslave, ifplugd, ifup, inetd, init, insmod, install, ionice, iostat, ip, ipaddr, ipcalc, ipcrm, ipcs, iplink, ipneigh,
iproute, iprule, iptunnel, kbd_mode, kill, killall, killall5, klogd, last, less, link, linux32, linux64, linuxrc, ln, loadfont,
loadkmap, logger, login, logname, logread, losetup, lpd, lpq, lpr, ls, lsattr, lsmod, lsof, lspci, lsscsi, lsusb, lzcat, lzma,
lzop, makedevs, makemime, man, md5sum, mdev, mesg, microcom, mim, mkdir, mkdosfs, mke2fs, mkfifo, mkfs.ext2, mkfs.minix, mkfs.vfat,
mknod, mkpasswd, mkswap, mktemp, modinfo, modprobe, more, mount, mountpoint, mpstat, mt, mv, nameif, nanddump, nandwrite,
nbd-client, nc, netstat, nice, nl, nmeter, nohup, nologin, nproc, nsenter, nslookup, ntpd, nuke, od, openvt, partprobe, passwd,
paste, patch, pgrep, pidof, ping, ping6, pipe_progress, pivot_root, pkill, pmap, popmaildir, poweroff, powertop, printenv, printf,
ps, pscan, pstree, pwd, pwdx, raidautorun, rdate, rdev, readahead, readlink, readprofile, realpath, reboot, reformime,
remove-shell, renice, reset, resize, resume, rev, rm, rmdir, rmmod, route, rpm, rpm2cpio, rtcwake, run-init, run-parts, runlevel,
runsv, runsvdir, rx, script, scriptreplay, sed, sendmail, seq, setarch, setconsole, setfattr, setfont, setkeycodes, setlogcons,
setpriv, setserial, setsid, setuidgid, sh, sha1sum, sha256sum, sha3sum, sha512sum, showkey, shred, shuf, slattach, sleep, smemcap,
softlimit, sort, split, ssl_client, start-stop-daemon, stat, strings, stty, su, sulogin, sum, sv, svc, svlogd, svok, swapoff,
swapon, switch_root, sync, sysctl, syslogd, tac, tail, tar, taskset, tc, tcpsvd, tee, telnet, telnetd, test, tftp, tftpd, time,
timeout, top, touch, tr, traceroute, traceroute6, true, truncate, ts, tty, ttysize, tunctl, ubiattach, ubidetach, ubimkvol,
ubirename, ubirmvol, ubirsvol, ubiupdatevol, udhcpc, udhcpc6, udhcpd, udpsvd, uevent, umount, uname, unexpand, uniq, unix2dos,
unlink, unlzma, unshare, unxz, unzip, uptime, users, usleep, uudecode, uuencode, vconfig, vi, vlock, volname, w, wall, watch,
watchdog, wc, wget, which, who, whoami, whois, xargs, xxd, xz, xzcat, yes, zcat, zcip
  \end{verbatim}
  \vfill
  \footnotesize
  Source: run \code{/bin/busybox} - July 2021 status
\end{frame}

\begin{frame}
  \frametitle{Configuring BusyBox}
  \begin{itemize}
  \item Get the latest stable sources from \url{https://busybox.net}
  \item Configure BusyBox (creates a \code{.config} file):
    \begin{itemize}
    \item \code{make defconfig}\\
      Good to begin with BusyBox.\\
      Configures BusyBox with all options for regular users.
    \item \code{make allnoconfig}\\
      Unselects all options. Good to configure only what you need.
    \end{itemize}
  \item \code{make menuconfig} (text)\\
    Same configuration interfaces as the ones used by the Linux kernel
    (though older versions are used, causing \code{make xconfig} to
    be broken in recent distros).
  \end{itemize}
\end{frame}

\begin{frame}
  \frametitle{BusyBox make menuconfig}
  \begin{columns}
    \column{0.5\textwidth}
    You can choose:
    \begin{itemize}
    \item the commands to compile,
    \item and even the command options and features that you need!
    \end{itemize}
    \column{0.5\textwidth}
    \includegraphics[width=\textwidth]{slides/sysdev-busybox/menuconfig-screenshot.png}
  \end{columns}
\end{frame}

\begin{frame}
  \frametitle{Compiling BusyBox}
  \begin{itemize}
  \item Set the cross-compiler prefix in the configuration interface: \\
    \code{Settings ->  Build Options ->  Cross Compiler
      prefix}\\
    Example: \code{arm-linux-}
  \item Set the installation directory in the configuration interface: \\
    \code{Settings ->  Installation Options}
    \code{  ->  Destination path for 'make install'}
  \item Add the cross-compiler path to the PATH environment variable:\\
    \code{export PATH=$HOME/x-tools/arm-unknown-linux-uclibcgnueabi/bin:$PATH}
  \item Compile BusyBox:\\
    \code{make}
  \item Install it (this creates a UNIX directory structure with symbolic
    links to the \code{busybox} executable):\\
    \code{make install}
  \end{itemize}
\end{frame}

\begin{frame}
  \frametitle{Applet highlight: BusyBox init}
  \begin{itemize}
  \item BusyBox provides an implementation of an \code{init} program
  \item Simpler than the init implementation found on desktop/server
    systems ({\em SysV init} or {\em systemd})
  \item A single configuration file: \code{/etc/inittab}
    \begin{itemize}
    \item Each line has the form \code{<id>::<action>:<process>}
    \end{itemize}
  \item Allows to start system services at startup, to control system
        shutdown, and to make sure that certain services are always
        running on the system.
  \item See \projfile{busybox}{examples/inittab} in BusyBox for details on the
    configuration
  \end{itemize}
\end{frame}

\begin{frame}
  \frametitle{Applet highlight: BusyBox vi}
  \begin{columns}
    \column{0.6\textwidth}
      \begin{itemize}
      \item If you are using BusyBox, adding \code{vi} support only adds
        about 20K
      \item You can select which exact features to compile in.
      \item Users hardly realize that they are using a lightweight \code{vi}
        version!
      \item Tip: you can learn \code{vi} on the desktop, by running the \code{vimtutor}
        command.
      \end{itemize}
    \column{0.4\textwidth}
      \includegraphics[width=\textwidth]{slides/sysdev-busybox/busybox-vi-configuration.png}
  \end{columns}
\end{frame}

\setuplabframe
{Tiny root filesystem built from scratch with BusyBox}
{
  \begin{itemize}
  \item Setting up a kernel to boot your system on a workstation
    directory exported by NFS
  \item Passing kernel command line parameters to boot on NFS
  \item Creating the full root filesystem from scratch.
    Populating it with BusyBox based utilities.
  \item System startup using BusyBox \code{init}
  \item Using the BusyBox HTTP server.
  \item Controlling the target from a web browser on the PC host.
  \item Setting up shared libraries on the target and compiling
    a sample executable.
  \end{itemize}
}
