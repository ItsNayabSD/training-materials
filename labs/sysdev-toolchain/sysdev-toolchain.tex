\subchapter{Building a cross-compiling toolchain}{Objective: Learn how
  to compile your own cross-compiling toolchain for the musl C
  library}

After this lab, you will be able to:

\begin{itemize}
\item Configure the {\em crosstool-ng} tool
\item Execute {\em crosstool-ng} and build up your own cross-compiling toolchain
\end{itemize}

\section{Setup}

Go to the \code{$HOME/__SESSION_NAME__-labs/toolchain} directory.

For this lab, you need a system or VM with a least 4 GB of RAM.

\section{Install needed packages}

Install the packages needed for this lab:

\begin{bashinput}
$ sudo apt install build-essential git autoconf bison flex texinfo help2man gawk libtool-bin libncurses5-dev unzip
\end{bashinput}

\section{Getting Crosstool-ng}

Let's download the sources of Crosstool-ng, through its git
source repository, and switch to a commit that we have tested:

\begin{bashinput}
$ git clone https://github.com/crosstool-ng/crosstool-ng
$ cd crosstool-ng/
$ git checkout crosstool-ng-1.26.0
\end{bashinput}

\section{Building and installing Crosstool-ng}

As we are not building Crosstool-ng from a release archive but from
a git repository, we first need to generate a \code{configure} script and
more generally all the generated files that are shipped in the source
archive for a release:

\begin{bashinput}
$ ./bootstrap
\end{bashinput}

We can then either install Crosstool-ng globally on the system, or keep it
locally in its download directory. We'll choose the latter
solution. As documented at
\url{https://crosstool-ng.github.io/docs/install/#hackers-way}, do:

\begin{bashinput}
$ ./configure --enable-local
$ make
\end{bashinput}

Then you can get Crosstool-ng help by running

\begin{bashinput}
$ ./ct-ng help
\end{bashinput}

\section{Configure the toolchain to produce}

A single installation of Crosstool-ng allows to produce as many
toolchains as you want, for different architectures, with different C
libraries and different versions of the various components.

Crosstool-ng comes with a set of ready-made configuration files for
various typical setups: Crosstool-ng calls them {\em samples}. They can be
listed by using \code{./ct-ng list-samples}.

We will load the
\ifdefstring{\labboard}{qemu}{Cortex A9}{}%
\ifdefstring{\labboard}{beaglebone}{Cortex A8}{}%
\ifdefstring{\labboard}{beagleplay}{\code{aarch64-unknown-linux-musl}}{}%
\ifdefstring{\labboard}{stm32mp1}{Cortex A5}{}
sample\ifdefstring{\labboard}{stm32mp1}{, as Crosstool-ng doesn't have
any sample for Cortex A7 yet}{}. Load it with the \code{./ct-ng} command.

Then, to refine the configuration, let's run the \code{menuconfig} interface:

\begin{bashinput}
$ ./ct-ng menuconfig
\end{bashinput}

In \code{Path and misc options}:
\begin{itemize}
\item If not set yet, enable \code{Try features marked as EXPERIMENTAL}
\item In some distributions, \code{wget2} is used instead of \code{wget}.
Since \code{wget2} doesn't support the passive-ftp option, you may need
to remove the \code{--passive-ftp} flags from the
\code{DOWNLOAD_WGET_OPTIONS}
\end{itemize}

\ifdefstring{\labboard}{stm32mp1}{
In \code{Target options}:
\begin{itemize}
\item Set \code{Emit assembly for CPU} (\code{ARCH_CPU}) to \code{cortex-a7}.
\item Set \code{Use specific FPU} (\code{ARCH_FPU}) to \code{vfpv4}.
\item Set \code{Floating point} to \code{hardware (FPU)}.
\end{itemize}
}{}

\ifdefstring{\labboard}{beaglebone}{
In \code{Target options}:
\begin{itemize}
\item Set \code{Use specific FPU} (\code{ARCH_FPU}) to \code{vfpv3}.
\item Set \code{Floating point} to \code{hardware (FPU)}.
\end{itemize}
}{}

\ifdefstring{\labboard}{beagleplay}{
In \code{Target options}:
\begin{itemize}
\item Set \code{Emit assembly for CPU} (\code{ARCH_CPU}) to \code{cortex-a53}.
\item Check that \code{Endianness} (\code{ARCH_ENDIAN}) is set to \code{Little endian}
\end{itemize}
}{}

In \code{Toolchain options}:
\ifdefstring{\labboard}{beagleplay}{
\begin{itemize}
\item Set \code{Tuple's vendor string} (\code{TARGET_VENDOR}) to \code{training}.
\item Set \code{Tuple's alias} (\code{TARGET_ALIAS}) to \code{aarch64-linux}.
      This way, we will be able to use the compiler as \code{aarch64-linux-gcc}
      instead of \code{aarch64-training-linux-musl-gcc}, which is
      much longer to type.
\end{itemize}
}{
\begin{itemize}
\item Set \code{Tuple's vendor string} (\code{TARGET_VENDOR}) to \code{training}.
\item Set \code{Tuple's alias} (\code{TARGET_ALIAS}) to \code{arm-linux}.
      This way, we will be able to use the compiler as \code{arm-linux-gcc}
      instead of \code{arm-training-linux-musleabihf-gcc}, which is
      much longer to type.
\end{itemize}
}

In \code{Operating System}:
\begin{itemize}
\item Set \code{Version of linux} to the closest, but older version to
      \workingkernel. It's important that the kernel headers used in
      the toolchain are not more recent than the kernel that will run
      on the board (v\workingkernel).
\end{itemize}

In \code{C-library}:
\begin{itemize}
  \item If not set yet, set \code{C library} to \code{musl}
        (\code{LIBC_MUSL})
  \item Keep the default version that is proposed
\end{itemize}

In \code{C compiler}:
\begin{itemize}
  \item Set \code{Version of gcc} to \code{13.2.0}.
  \item Make sure that \code{C++} (\code{CC_LANG_CXX}) is enabled
\end{itemize}

In \code{Debug facilities}:
\begin{itemize}
\item Remove all options here. Some debugging tools can be provided
      in the toolchain, but they can also be built by filesystem
      building tools.
\end{itemize}

Explore the different other available options by traveling through the
menus and looking at the help for some of the options. Don't hesitate
to ask your trainer for details on the available options. However,
remember that we tested the labs with the configuration described
above. You might waste time with unexpected issues if you customize the
toolchain configuration.

\section{Produce the toolchain}

Nothing is simpler:

\begin{bashinput}
$ ./ct-ng build
\end{bashinput}

The toolchain will be installed by default in \code{$HOME/x-tools/}.
That's something you could have changed in Crosstool-ng's configuration.

And wait!

\section{Testing the toolchain}

You can now test your toolchain by adding
\ifdefstring{\labboard}{beagleplay}{\code{$HOME/x-tools/aarch64-training-linux-musl/bin/}}
{\code{$HOME/x-tools/arm-training-linux-musleabihf/bin/}} to your
\code{PATH} environment variable and compiling the simple
\code{hello.c} program in your main lab directory with
\ifdefstring{\labboard}{beagleplay}{\code{aarch64-linux-gcc}}{\code{arm-linux-gcc}}:

\begin{bashinput}
$ %\ifdefstring{\labboard}{beagleplay}{aarch64-linux-gcc}{arm-linux-gcc}% -o hello hello.c
\end{bashinput}
You can use the \code{file} command on your binary to make sure it has
correctly been compiled for the \ifdefstring{\labboard}{beagleplay}{AARCH64}{ARM}
architecture.

Did you know that you can still execute this binary from your x86 host?
To do this, install the QEMU user emulator, which just emulates target
instruction sets, not an entire system with devices:

\begin{bashinput}
$ sudo apt install qemu-user
\end{bashinput}

Now, try to run QEMU \ifdefstring{\labboard}{beagleplay}{AARCH64}{ARM}
user emulator:

\if\defstring{\labboard}{beagleplay}
\begin{bashinput}
$ qemu-aarch64 hello
qemu-aarch64: Could not open '/lib/ld-musl-aarch64.so.1': No such file or directory
\end{bashinput}
\else
\begin{bashinput}
$ qemu-arm hello
qemu-arm: Could not open '/lib/ld-musl-armhf.so.1': No such file or directory
\end{bashinput}
\fi

What's happening is that \ifdefstring{\labboard}{beagleplay}{\code{qemu-aarch64}}
{\code{qemu-arm}} is missing the shared library loader
(compiled for \ifdefstring{\labboard}{beagleplay}{AARCH64}{ARM}) that this
binary relies on. Let's find it in our newly compiled toolchain:

\begin{bashinput}
$ find ~/x-tools -name %\ifdefstring{\labboard}{beagleplay}{ld-musl-aarch64.so.1}{ld-musl-armhf.so.1}%
\end{bashinput}

\if\defstring{\labboard}{beagleplay}
\begin{terminaloutput}
/home/tux/x-tools/aarch64-training-linux-musl/aarch64-training-linux-musl/sysroot/lib/ld-musl-aarch64.so.1
\end{terminaloutput}
\else
\begin{terminaloutput}
/home/tux/x-tools/arm-training-linux-musleabihf/arm-training-linux-musleabihf/sysroot/lib/ld-musl-armhf.so.1
\end{terminaloutput}
\fi

We can now use the \code{-L} option of \ifdefstring{\labboard}{beagleplay}
{\code{qemu-aarch64}}{\code{qemu-arm}} to let it know where shared libraries are:

\if\defstring{\labboard}{beagleplay}
\begin{bashinput}
$ qemu-aarch64 -L ~/x-tools/aarch64-training-linux-musl/aarch64-training-linux-musl/sysroot hello
\end{bashinput}
\else
\begin{bashinput}
$ qemu-arm -L ~/x-tools/arm-training-linux-musleabihf/arm-training-linux-musleabihf/sysroot hello
\end{bashinput}
\fi

\begin{terminaloutput}
Hello world!
\end{terminaloutput}

\section{Cleaning up}

{\em Do this only if you have limited storage space. In case you made a
mistake in the toolchain configuration, you may need to run Crosstool-ng
again, keeping generated files would save a significant amount of time.}

To save about 9 GB of storage space, do a \code{./ct-ng clean} in the
Crosstool-NG source directory. This will remove the source code of the
different toolchain components, as well as all the generated files
that are now useless since the toolchain has been installed in
\code{$HOME/x-tools}.
