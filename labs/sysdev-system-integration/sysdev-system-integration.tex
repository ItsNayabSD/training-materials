\subchapter{System Integration - Using systemd}
  {Objectives: Get familiar with the {\em systemd} init system.}

\section{Goals}

Compared to the previous lab, we go on increasing the complexity
of the system, this time by using the {\em systemd} init system,
and by taking advantage of it to add a few extra features, in particular
ones that will be useful for debugging in the next lab.

\section{Setup}

Since {\em systemd} requires the GNU C library, we are going
to make a new Buildroot build in a new working directory, and
using a different cross-compiling toolchain.

So, create the \code{$HOME/__SESSION_NAME__-labs/integration} directory
and go inside it.

Make a new clone of Buildroot from the existing local Git repository,
and checkout our \code{bootlin-labs} branch:

\begin{bashinput}
git clone $HOME/__SESSION_NAME__-labs/buildroot/buildroot
cd buildroot
git checkout bootlin-labs
\end{bashinput}

\section{Root filesystem overlay}

Remove \code{etc/init.d/} from the root filesystem overlay.
It was adapted to {\em BusyBox init}, not to {\em systemd}:

\begin{bashinput}
rm -r board/bootlin/training/rootfs-overlay/etc/init.d/
\end{bashinput}

\section{Buildroot configuration}

Configure Buildroot as follows:

\begin{itemize}
\item \code{Target options}
  \begin{itemize}
  \item Select the same architecture and CPU settings as in
        the previous lab.
  \end{itemize}
\item \code{Toolchain}
  \begin{itemize}
  \item \code{Toolchain type}: \code{External toolchain}
  \item \code{Toolchain}: \code{Bootlin toolchains}\\
        This time, we will use a Bootlin ready-made toolchain for {\em
        glibc}, as this is necessary for using {\em systemd}.
  \item \code{Toolchain origin}: \code{Toolchain to be downloaded and installed}
  \item \code{Bootlin toolchain variant}: \ifdefstring{\labboard}{beagleplay}
        {\code{aarch64 glibc bleeding-edge 2022.08-1}}{\code{armv7-eabihf glibc bleeding-edge}}
  \item Select \code{Copy gdb server to the Target}
  \end{itemize}
\item \code{System configuration}
  \begin{itemize}
  \item \code{Init system}: \code{systemd}
  \item \code{Root filesystem overlay directories}: \code{board/bootlin/training/rootfs-overlay}
  \end{itemize}
\item \code{Kernel}
  \begin{itemize}
  \item Enable \code{Linux Kernel}
  \if\defstring{\labboard}{beagleplay}
  \item Set \code{Kernel version} to \code{Custom version}
  \item Set \code{Kernel version} to \code{6.6.20}
  \else
  \item Set \code{Kernel version} to \code{Custom version}
  \item Set \code{Kernel version} to your kernel version. You can use \code{make kernelversion} to
  get it from the Linux kernel source tree.
  \fi
  \item Set \code{Custom kernel patches} to \code{board/bootlin/training/0001-Custom-DTS-for-Bootlin-lab.patch}
  \item Set \code{Kernel configuration} to \code{Using a custom (def)config file})
  \item Set \code{Configuration file path} to \code{board/bootlin/training/linux.config}
  \item Select \code{Build a Device Tree Blob (DTB)}
  \item Set \code{In-tree Device Tree Source file names} to
        \ifdefstring{\labboard}{stm32mp1}{\code{st/stm32mp157a-dk1-custom}}{}
        \ifdefstring{\labboard}{beaglebone}{\code{ti/omap/am335x-boneblack-custom}}{}
        \ifdefstring{\labboard}{beagleplay}{\code{ti/k3-am625-beagleplay-custom}}{}
  \ifdefstring{\labboard}{beagleplay}{\item Set \code{Kernel Binary format} to \code{Image.gz}}{}
  \end{itemize}
\item \code{Target packages}
  \begin{itemize}
  \item \code{Audio and video applications}
    \begin{itemize}
    \item We won't need \code{alsa-utils} this time.
    \item Select \code{mpd}, and in the submenu:
      \begin{itemize}
           \item Keep only \code{alsa}, \code{vorbis} and \code{tcp sockets}
      \end{itemize}
    \item Select \code{mpd-mpc}.
    \end{itemize}
  \item \code{Hardware handling}
    \begin{itemize}
    \item Select \code{nunchuk driver}
    \end{itemize}
  \item \code{Networking applications}
    \begin{itemize}
    \item Select \code{dropbear}, a lightweight SSH server used instead
          of OpenSSH in most embedded devices. You don't need to enable
          client support (building an SSH client).
    \end{itemize}
  \end{itemize}
\item \code{Filesystem images}
  \begin{itemize}
     \item Select \code{tar the root filesystem}
  \end{itemize}
\end{itemize}

\section{Build and test the new system}

Now build the full system.

Once the build is over, generate the dependency graph again and find out
the new dependencies introduced by using {\em systemd}.

To test the new system, create a new \code{nfsroot} directory, extract
the new root filesystem into it, and boot your board on it through NFS.

You should see the system booting through {\em systemd}, with all the
{\em systemd} targets and system services starting one by one, with a
total boot time which looks slower than before. That's because the
system configuration is more complex, but also more versatile, being
ready to run more complex services and applications.

You can ask {\em systemd} to show you the various services which were
started:

\bashcmd{# systemctl status}

You can also check all the mounted filesystems and be impressed:

\bashcmd{# mount}

\section{Inspecting the system}

On the target, look at the contents of \code{/lib/systemd}. You will
see the implementation of most {\em systemd} targets and services.

In particular, check out \code{/lib/systemd/user/} containing
some unnecessary targets in our case such as \code{bluetooth.target}.

However, check the \code{mpd.service} file for our MPD server.
This should help you to realize all the options provided
by {\em systemd} to start and control system services, while
keeping the system secure and their resources under control.

You won't be able to match this level of control and security in a
"hand-made" system.

\section{Understanding automatic module loading with Udev}

Check the currently loaded modules on your system. Surprise: both
the Nunchuk and USB audio modules are already loaded.
We didn't have anything to set up and {\em systemd} automatically loaded
the modules associated to connected hardware.

Let's find out why...

On the target, go to \code{/lib/udev/rules.d}. You will find all the
standard rules for {\em Udev}, the part of {\em systemd} which handles
hardware events, takes care of the permissions and ownership of device
files, notifies other userspace programs, and among others, load
kernel modules.

Open \code{80-drivers.rules}, which is the rule allowing {\em Udev}
to load kernel modules for detected devices. Here is its most important
line:

\begin{verbatim}
ENV{MODALIAS}=="?*", RUN{builtin}+="kmod load '$env{MODALIAS}'"
\end{verbatim}

This is when the \code{modules.alias} file comes into play.
When a new device is found, the kernel passes a \code{MODALIAS}
environment variable to {\em Udev}, containing which bus this
happened on and the attributes of the device on this bus.
Thanks to the module aliases, the right module gets loaded.
We already explained that in the lectures when talking about
the output of \code{make modules_install}.

Find where the \code{modules.alias} file is located and
you will find the two lines that allowed to load our
\code{snd_usb_audio} and \code{nunchuk} modules:

\begin{verbatim}
...
alias usb:v*p*d*dc*dsc*dp*ic01isc01ip*in* snd_usb_audio
alias usb:v2B53p0031d*dc*dsc*dp*ic*isc*ip*in* snd_usb_audio
...
alias of:N*T*Cnintendo,nunchuk nunchuk

\end{verbatim}

For \code{snd_usb_audio}, there are many possible matching values,
so it's not straightforward to be sure which matched your particular
device.

However, you can find in {\em sysfs} which \code{MODALIAS}
was emitted for your device:

\begin{bashinput}
# cd /sys/class/sound/card0/device
# ls -la
# cat modalias
usb:v1B3Fp2008d0100dc00dsc00dp00ic01isc01ip00in00
\end{bashinput}

With a bit of patience, you could find the matching line in the
\code{modules.alias} file.

If you want to see the information sent to {\em Udev} by the
kernel when a new device is plugged in, here are a few debugging
commands.

First unplug your device and run:

\bashcmd{# udevadm monitor}

Then plug in your headset again. You will find all the events emitted
by the kernel, and with the same string (with \code{UDEV} instead of
\code{KERNEL}), the time when {\em Udev} finished processing each event.

You can also see the \code{MODALIAS} values carried by these events:

\bashcmd{# udevadm monitor --env}

As far as the Nunchuk is concerned, we cannot easily remove it from the
Device Tree and add it back, but it's easier to find its \code{MODALIAS}
value:

\begin{bashinput}
# cd /sys/bus/i2c/devices
# ls -la
\end{bashinput}

Here you will recognize our Nunchuk device through its \code{0x52}
address.

\begin{bashinput}
# cd %\ifdefstring{\labboard}{beagleplay}{3}{1}%-0052
# ls -la
# cat modalias
of:NjoystickT(null)Cnintendo,nunchuk
\end{bashinput}

Here the bus is \code{of}, meaning {\em Open Firmware}, which
was the former name of the Device Tree. When an event was emitted by
the kernel with this \code{MODALIAS} string, the \code{nunchuk} module
got loaded by {\em Udev} thanks to the matching alias.

This actually happened when {\em systemd} ran the {\em coldplugging}
operation: at system startup, it asked the kernel to emit hotplug events
for devices already present when the system booted:

\begin{verbatim}
[  OK  ] Finished Coldplug All udev Devices.
\end{verbatim}

On non-x86 platforms, that's typically for devices described in the
Device Tree. This way, both {\em static} and {\em hotplugged} devices
can be handled in the same way, using the same {\em Udev} rules.

\section{Testing your system}

Make sure that audio playback still works on your system:

\begin{bashinput}
# mpc update
# mpc add /
# mpc play
\end{bashinput}

If it doesn't, look at the {\em systemd} logs in your serial console
history. {\em systemd} should let you know about the failing services
and the commands to run to get more details.
