\section{ASoC}

\begin{frame}{ASoC}
  ASoC, ALSA System on Chip: is a Linux kernel subsystem created to
  provide better ALSA support for system-on-chip and portable audio
  codecs. It allows to reuse codec drivers across multiple
  architectures and provides an API to integrate them with the SoC
  audio interface.
  \begin{itemize}
  \item created for that use case
  \item designed for codec drivers reuse
  \item has an API to write codec drivers
  \item has an API to write SoC interface drivers
  \end{itemize}
\end{frame}

\begin{frame}{ASoC components}
  \begin{itemize}
  \item Codec class drivers: define the codec capabilities (audio
    interface, audio controls, analog inputs and outputs).
  \item Platform class drivers: defines the SoC audio interface (also
    referred as CPU DAI), sets up DMA when applicable.
  \item Codec to platform integration: nowadays, usually done through
    device tree, previously required writing a machine driver in C.
  \end{itemize}
  Note: The codec can be part of another IC (PMIC, Bluetooth or MODEM
  chips).
\end{frame}

\subsection{simple-audio-card}

\begin{frame}[fragile]{simple-card}
  Most sound cards, can now be described using device tree. This is
  done using a sound node with a \code{simple-audio-card} compatible
  string.
  \begin{itemize}
  \item The DT bindings are documented in
    \kfile{Documentation/devicetree/bindings/sound/simple-card.yaml}
  \item The driver handling it is \kfile{sound/soc/generic/simple-card.c}
  \end{itemize}
  Since 2017, OF-graph based bindings are available.
  \begin{itemize}
  \item They are documented in
    \kfile{Documentation/devicetree/bindings/sound/audio-graph-card.yaml}
  \item The driver handling it is \kfile{sound/soc/generic/audio-graph-card.c}
  \end{itemize}
  Both required a few changes in the SoC DAI drivers to be usable for
  example to select the audio mode for the SSC on Microchip SoCs or
  configure properly the i.MX audmux.
\end{frame}

\begin{frame}[fragile]{simple-card - example 1}
  Let's say we have an ADAU1372 codec connected to an i.Mx6UL SAI.
  First, enable the SAI and the codec:
  \begin{block}{}
    \fontsize{7}{6}\selectfont
    \begin{minted}{dts}
&sai2 {
        pinctrl-names = "default";
        pinctrl-0 = <&pinctrl_sai2>;
        status = "okay";
};

&i2c1 {
        adau1372: codec@3c {
                #sound-dai-cells = <0>;
                compatible = "adi,adau1372";
                reg = <0x3c>;
                clock-names = "mclk";
                clocks = <&adau1372z_xtal>;
        };
};

/ {
        adau1372z_xtal: adau1372z_xtal {
                compatible = "fixed-clock";
                #clock-cells = <0>;
                clock-frequency = <12288000>;
        };
};
    \end{minted}
  \end{block}
\end{frame}

\begin{frame}[fragile]{simple-card - example 1}
  Now, describe the sound card:
  \begin{block}{}
    \fontsize{7}{6}\selectfont
    \begin{minted}{dts}
        sound {
                compatible = "simple-audio-card";
                simple-audio-card,name = "imx6ul-adau1372";

                simple-audio-card,dai-link@0 {
                        format = "i2s";
                        bitclock-master = <&adau1372_dai>;
                        frame-master = <&adau1372_dai>;

                        sai2_dai: cpu {
                                sound-dai = <&sai2>;
                        };

                        adau1372_dai: codec {
                                sound-dai = <&adau1372>;
                        };
                };
        };
    \end{minted}
  \end{block}
  For convenience, the codec is the producer, it generates both BCLK and
  FSCLK.
\end{frame}

\begin{frame}[fragile]{simple-card - example 2}
  The ADAU1372 has actually 4 channels and can do TDM:
  \begin{block}{}
    \fontsize{7}{6}\selectfont
    \begin{minted}{dts}
        sound {
                compatible = "simple-audio-card";
                simple-audio-card,name = "imx6ul-adau1372";

                simple-audio-card,dai-link@0 {
                        format = "i2s";
                        bitclock-master = <&adau1372_dai>;
                        frame-master = <&adau1372_dai>;

                        sai2_dai: cpu {
                                sound-dai = <&sai2>;
                                dai-tdm-slot-num = <4>;
                                dai-tdm-slot-width = <32>;
                        };

                        adau1372_dai: codec {
                                sound-dai = <&adau1372>;
                                dai-tdm-slot-num = <4>;
                                dai-tdm-slot-width = <32>;
                        };
                };
        };
    \end{minted}
  \end{block}
\end{frame}

\begin{frame}[fragile]{simple-card - example 3}
  However, the ADAU1372 has an hardware issue and doesn't generate the
  proper BCLK when doing TDM4 with a 32kHz sample rate. The SAI has to
  be master:
  \begin{block}{}
    \fontsize{7}{6}\selectfont
    \begin{minted}{dts}
        sound {
                compatible = "simple-audio-card";
                simple-audio-card,name = "imx6ul-adau1372";

                simple-audio-card,dai-link@0 {
                        format = "i2s";
                        bitclock-master = <&sai2_dai>;
                        frame-master = <&sai2_dai>;

                        sai2_dai: cpu {
                                sound-dai = <&sai2>;
                                dai-tdm-slot-num = <4>;
                                dai-tdm-slot-width = <32>;
                        };

                        adau1372_dai: codec {
                                sound-dai = <&adau1372>;
                                dai-tdm-slot-num = <4>;
                                dai-tdm-slot-width = <32>;
                        };
                };
        };
    \end{minted}
  \end{block}
\end{frame}

\begin{frame}[fragile]{simple-card - example 3}
  The result is not what is expected:
  \begin{block}{}
    \fontsize{7}{6}\selectfont
    \begin{minted}{console}
    # aplay test.wav
    Playing WAVE 'test.wav' :
    Signed 16 bit Little Endian, Rate  32000 Hz, Stereo
    aplay: set_params:1403: Unable to install hw params:
    [...]
    # dmesg
    [...]
    fsl-sai 202c000.sai: failed to derive required Tx rate: 4096000
    fsl-sai 202c000.sai: ASoC: can't set 202c000.sai hw params: -22
    # cat /sys/kernel/debug/clk/clk_summary
    pll3                              1        1        0   480000000          0     0  50000
       pll3_bypass                    1        1        0   480000000          0     0  50000
          pll3_usb_otg                2        3        0   480000000          0     0  50000
             pll3_pfd2_508m           0        0        0   508235294          0     0  50000
                sai2_sel              0        0        0   508235294          0     0  50000
                   sai2_pred          0        0        0   127058824          0     0  50000
                      sai2_podf       0        0        0    63529412          0     0  50000
                         sai2         0        0        0    63529412          0     0  50000
    \end{minted}
  \end{block}
  Indeed, there is no way for the SAI to divide 63529412 to get the
  proper BCLK!
\end{frame}

\begin{frame}[fragile]{device tree - clocks}
  It is possible to reparent clocks using
  \code{assigned-clock-parents} and set the clock rate using
  \code{assigned-clock-rates}.
  \begin{block}{}
    \fontsize{7}{6}\selectfont
    \begin{minted}{dts}
&sai2 {
        pinctrl-names = "default";
        pinctrl-0 = <&pinctrl_sai2>;
        assigned-clocks = <&clks IMX6UL_CLK_SAI2_SEL>, <&clks IMX6UL_CLK_SAI2>;
        assigned-clock-parents = <&clks IMX6UL_CLK_PLL4_AUDIO_DIV>;
        assigned-clock-rates = <196608000>, <24576000>;
        status = "okay";
};
    \end{minted}
  \end{block}
  Notice that 24.576MHz was selected for the sai input clock as it is
  not able to divide by 3 to obtain the 4.096MHz BCLK.
\end{frame}

\begin{frame}[fragile]{simple-card - example 4}
  There is a possible cost reduction, the SAI is able to output its
  clock to feed to the codec MCLK instead of the crystal:
  \begin{block}{}
    \fontsize{7}{6}\selectfont
    \begin{minted}{dts}
&sai2 {
        pinctrl-names = "default";
        pinctrl-0 = <&pinctrl_sai2>;
        fsl,sai-mclk-direction-output;
        status = "okay";
};

&i2c1 {
        adau1372: codec@3c {
                #sound-dai-cells = <0>;
                compatible = "adi,adau1372";
                reg = <0x3c>;
                clock-names = "mclk";
                clocks = <&clks IMX6UL_CLK_SAI2>;
                assigned-clocks = <&clks IMX6UL_CLK_SAI2_SEL>, <&clks IMX6UL_CLK_SAI2>;
                assigned-clock-parents = <&clks IMX6UL_CLK_PLL4_AUDIO_DIV>;
                assigned-clock-rates = <196608000>, <24576000>;
        };
};
    \end{minted}
  \end{block}
  This replaces the 12.288MHz crystal by the 24.576 MCLK from the SAI.
  This works because the codec has a configurable divider for MCLK and
  can divide by 2. Also the clock parents and rates assignment has
  moved to the codec because of probing order.
\end{frame}

\begin{frame}[fragile]{simple-card - routing}
  It is possible but not mandatory to list the actual audio
  connections present on the board, this is called routing.
  The first step is to define the board connectors, in this case two
  stereo line input jack (Line0 and Line1) and a stereo jack output.
  \begin{block}{}
    \fontsize{7}{6}\selectfont
    \begin{minted}{dts}
        simple-audio-card,widgets =
                "Line", "Line0",
                "Line", "Line1",
                "Headphone", "Headphone Jack",
    \end{minted}
  \end{block}
\end{frame}

\begin{frame}[fragile]{simple-card - routing}
  Routing audio from the codec to the board connector is then done
  using \code{simple-audio-card,routing}
  \begin{block}{}
    \fontsize{7}{6}\selectfont
    \begin{minted}{dts}
        simple-audio-card,routing =
                "AIN0", "Line0",
                "AIN1", "Line0",
                "AIN2", "Line1",
                "AIN3", "Line1",
                "Headphone Jack", "HPOUTL",
                "Headphone Jack", "HPOUTR",
    \end{minted}
  \end{block}
  Look for \code{SND_SOC_DAPM_OUTPUT} and \code{SND_SOC_DAPM_INPUT} to
  know what the codec is providing.
\end{frame}

\subsection{Machine driver}

\begin{frame}[fragile]{Machine driver}
The machine driver registers a \kstruct{snd_soc_card}.
      \begin{block}{\kfile{include/sound/soc.h}}
        \fontsize{8}{8}\selectfont
        \begin{minted}{c}
int snd_soc_register_card(struct snd_soc_card *card);
int snd_soc_unregister_card(struct snd_soc_card *card);
int devm_snd_soc_register_card(struct device *dev, struct snd_soc_card *card);
[...]
/* SoC card */
struct snd_soc_card {
        const char *name;
        const char *long_name;
        const char *driver_name;
        struct device *dev;
        struct snd_card *snd_card;
[...]
        /* CPU <--> Codec DAI links  */
        struct snd_soc_dai_link *dai_link;  /* predefined links only */
        int num_links;  /* predefined links only */
        struct list_head dai_link_list; /* all links */
        int num_dai_links;
[...]
};
        \end{minted}
      \end{block}
\end{frame}

\begin{frame}[fragile]{\kstruct{snd_soc_dai_link}}
\kstruct{snd_soc_dai_link} is used to create the link between the
  CPU DAI and the codec DAI.
  \begin{block}{\kfile{include/sound/soc.h}}
    \fontsize{8}{8}\selectfont
    \begin{minted}{c}
struct snd_soc_dai_link {
        /* config - must be set by machine driver */
        const char *name;                        /* Codec name */
        const char *stream_name;                /* Stream name */

        /*
         * You MAY specify the link's CPU-side device, either by device name,
         * or by DT/OF node, but not both. If this information is omitted,
         * the CPU-side DAI is matched using .cpu_dai_name only, which hence
         * must be globally unique. These fields are currently typically used
         * only for codec to codec links, or systems using device tree.
         */
        /*
         * You MAY specify the DAI name of the CPU DAI. If this information is
         * omitted, the CPU-side DAI is matched using .cpu_name/.cpu_of_node
         * only, which only works well when that device exposes a single DAI.
         */
        struct snd_soc_dai_link_component *cpus;
        unsigned int num_cpus;
     \end{minted}
   \end{block}
\end{frame}

\begin{frame}[fragile]{\kstruct{snd_soc_dai_link}}
  \begin{block}{}
    \fontsize{10}{10}\selectfont
    \begin{minted}{c}
        /*
         * You MUST specify the link's codec, either by device name, or by
         * DT/OF node, but not both.
         */
        /* You MUST specify the DAI name within the codec */
        struct snd_soc_dai_link_component *codecs;
        unsigned int num_codecs;
[...]
        unsigned int dai_fmt;           /* format to set on init */
[...]
}
    \end{minted}
  \end{block}
\end{frame}

\begin{frame}[fragile]{Example 1}
  \begin{block}{\kfile{sound/soc/atmel/atmel_wm8904.c}}
    \fontsize{8}{8}\selectfont
    \begin{minted}{c}
SND_SOC_DAILINK_DEFS(pcm,
        DAILINK_COMP_ARRAY(COMP_EMPTY()),
        DAILINK_COMP_ARRAY(COMP_CODEC(NULL, "wm8904-hifi")),
        DAILINK_COMP_ARRAY(COMP_EMPTY()));

static struct snd_soc_dai_link atmel_asoc_wm8904_dailink = {
        .name = "WM8904",
        .stream_name = "WM8904 PCM",
        .dai_fmt = SND_SOC_DAIFMT_I2S
                | SND_SOC_DAIFMT_NB_NF
                | SND_SOC_DAIFMT_CBP_CFP,
        .ops = &atmel_asoc_wm8904_ops,
        SND_SOC_DAILINK_REG(pcm),
};

static struct snd_soc_card atmel_asoc_wm8904_card = {
        .name = "atmel_asoc_wm8904",
        .owner = THIS_MODULE,
        .dai_link = &atmel_asoc_wm8904_dailink,
        .num_links = 1,
        .dapm_widgets = atmel_asoc_wm8904_dapm_widgets,
        .num_dapm_widgets = ARRAY_SIZE(atmel_asoc_wm8904_dapm_widgets),
        .fully_routed = true,
};
    \end{minted}
  \end{block}
\end{frame}

\begin{frame}[fragile]{Example 1}
  \begin{block}{\kfile{sound/soc/atmel/atmel_wm8904.c}}
    \fontsize{8}{8}\selectfont
    \begin{minted}{c}
static int atmel_asoc_wm8904_dt_init(struct platform_device *pdev)
{
        struct device_node *np = pdev->dev.of_node;
        struct device_node *codec_np, *cpu_np;
        struct snd_soc_card *card = &atmel_asoc_wm8904_card;
        struct snd_soc_dai_link *dailink = &atmel_asoc_wm8904_dailink;
[...]
        cpu_np = of_parse_phandle(np, "atmel,ssc-controller", 0);
        if (!cpu_np) {
                dev_err(&pdev->dev, "failed to get dai and pcm info\n");
                ret = -EINVAL;
                return ret;
        }
        dailink->cpus->of_node = cpu_np;
        dailink->platforms->of_node = cpu_np;
        of_node_put(cpu_np);

        codec_np = of_parse_phandle(np, "atmel,audio-codec", 0);
        if (!codec_np) {
                dev_err(&pdev->dev, "failed to get codec info\n");
                ret = -EINVAL;
                return ret;
        }
        dailink->codecs->of_node = codec_np;
        of_node_put(codec_np);
    \end{minted}
  \end{block}
\end{frame}

\begin{frame}[fragile]{Example 1}
  \begin{block}{\kfile{sound/soc/atmel/atmel_wm8904.c}}
    \fontsize{8}{8}\selectfont
    \begin{minted}{c}
static int atmel_asoc_wm8904_probe(struct platform_device *pdev)
{
        struct snd_soc_card *card = &atmel_asoc_wm8904_card;
        struct snd_soc_dai_link *dailink = &atmel_asoc_wm8904_dailink;
        int id, ret;

        card->dev = &pdev->dev;
        ret = atmel_asoc_wm8904_dt_init(pdev);
        if (ret) {
                dev_err(&pdev->dev, "failed to init dt info\n");
                return ret;
        }

        id = of_alias_get_id((struct device_node *)dailink->cpus->of_node, "ssc");
        ret = atmel_ssc_set_audio(id);
        if (ret != 0) {
                dev_err(&pdev->dev, "failed to set SSC %d for audio\n", id);
                return ret;
        }

        ret = snd_soc_register_card(card);
        if (ret) {
                dev_err(&pdev->dev, "snd_soc_register_card failed\n");
                goto err_set_audio;
        }
[...]
}
    \end{minted}
  \end{block}
\end{frame}

\begin{frame}[fragile]{Routing}
  After linking the codec driver with the SoC DAI driver, it is still
  necessary to define what are the codec outputs and inputs that are
  actually used on the board. This is called routing.

  \begin{itemize}
  \item statically: using the \code{.dapm_routes} and
    \code{.num_dapm_routes} members of \kstruct{snd_soc_card}
  \item from device tree:
    \begin{minted}{c}
    int snd_soc_of_parse_audio_routing(struct snd_soc_card *card,
                                       const char *propname);
    \end{minted}
  \end{itemize}
\end{frame}

\begin{frame}[fragile]{Routing example: static}
  \begin{block}{\kfile{sound/soc/rockchip/rockchip_max98090.c}}
    \fontsize{8}{8}\selectfont
    \begin{minted}{c}
static const struct snd_soc_dapm_route rk_audio_map[] = {
        {"IN34", NULL, "Headset Mic"},
        {"IN34", NULL, "MICBIAS"},
        {"Headset Mic", NULL, "MICBIAS"},
        {"DMICL", NULL, "Int Mic"},
        {"Headphone", NULL, "HPL"},
        {"Headphone", NULL, "HPR"},
        {"Speaker", NULL, "SPKL"},
        {"Speaker", NULL, "SPKR"},
};
[...]
static struct snd_soc_card snd_soc_card_rk = {
        .name = "ROCKCHIP-I2S",
        .owner = THIS_MODULE,
        .dai_link = &rk_dailink,
        .num_links = 1,
[...]
        .dapm_widgets = rk_dapm_widgets,
        .num_dapm_widgets = ARRAY_SIZE(rk_dapm_widgets),
        .dapm_routes = rk_audio_map,
        .num_dapm_routes = ARRAY_SIZE(rk_audio_map),
        .controls = rk_mc_controls,
        .num_controls = ARRAY_SIZE(rk_mc_controls),
};
    \end{minted}
  \end{block}
\end{frame}

\begin{frame}[fragile]{Routing example: DT}
  \begin{block}{\kfile{sound/soc/atmel/atmel_wm8904.c}}
    \fontsize{10}{10}\selectfont
    \begin{minted}{c}

static int atmel_asoc_wm8904_dt_init(struct platform_device *pdev)
{
[...]
        ret = snd_soc_of_parse_card_name(card, "atmel,model");
        if (ret) {
                dev_err(&pdev->dev, "failed to parse card name\n");
                return ret;
        }

        ret = snd_soc_of_parse_audio_routing(card, "atmel,audio-routing");
        if (ret) {
                dev_err(&pdev->dev, "failed to parse audio routing\n");
                return ret;
        }
[...]
}
    \end{minted}
  \end{block}
\end{frame}

\begin{frame}[fragile]{Routing example: DT}
  \begin{block}{\kfile{Documentation/devicetree/bindings/sound/atmel-wm8904.txt}}
    \fontsize{8}{8}\selectfont
    \begin{minted}{c}
  - atmel,audio-routing: A list of the connections between audio components.
    Each entry is a pair of strings, the first being the connection's sink,
    the second being the connection's source. Valid names for sources and
    sinks are the WM8904's pins, and the jacks on the board:

    WM8904 pins:

    * IN1L
    * IN1R
    * IN2L
    * IN2R
    * IN3L
    * IN3R
    * HPOUTL
    * HPOUTR
    * LINEOUTL
    * LINEOUTR
    * MICBIAS

    Board connectors:

    * Headphone Jack
    * Line In Jack
    * Mic
    \end{minted}
  \end{block}
\end{frame}

\begin{frame}[fragile]{Routing example}
  \begin{block}{\kfile{Documentation/devicetree/bindings/sound/atmel-wm8904.txt}}
    \fontsize{8}{8}\selectfont
    \begin{minted}{c}
Example:
sound {
        compatible = "atmel,asoc-wm8904";
        pinctrl-names = "default";
        pinctrl-0 = <&pinctrl_pck0_as_mck>;

        atmel,model = "wm8904 @ AT91SAM9N12EK";

        atmel,audio-routing =
                "Headphone Jack", "HPOUTL",
                "Headphone Jack", "HPOUTR",
                "IN2L", "Line In Jack",
                "IN2R", "Line In Jack",
                "Mic", "MICBIAS",
                "IN1L", "Mic";

        atmel,ssc-controller = <&ssc0>;
        atmel,audio-codec = <&wm8904>;
};
    \end{minted}
  \end{block}
\end{frame}

\begin{frame}[fragile]{Routing: codec pins}
  The available codec pins are defined in the codec driver. Look for
  the \code{SND_SOC_DAPM_INPUT} and \code{SND_SOC_DAPM_OUTPUT}
  definitions.

  \begin{block}{\kfile{sound/soc/codecs/wm8904.c}}
    \fontsize{8}{8}\selectfont
    \begin{minted}{c}
static const struct snd_soc_dapm_widget wm8904_adc_dapm_widgets[] = {
SND_SOC_DAPM_INPUT("IN1L"),
SND_SOC_DAPM_INPUT("IN1R"),
SND_SOC_DAPM_INPUT("IN2L"),
SND_SOC_DAPM_INPUT("IN2R"),
SND_SOC_DAPM_INPUT("IN3L"),
SND_SOC_DAPM_INPUT("IN3R"),
[...]
};

static const struct snd_soc_dapm_widget wm8904_dac_dapm_widgets[] = {
[...]
SND_SOC_DAPM_OUTPUT("HPOUTL"),
SND_SOC_DAPM_OUTPUT("HPOUTR"),
SND_SOC_DAPM_OUTPUT("LINEOUTL"),
SND_SOC_DAPM_OUTPUT("LINEOUTR"),
};
    \end{minted}
  \end{block}
\end{frame}

\begin{frame}[fragile]{Routing: board connectors}
  The board connectors are defined in the machine driver, in the
  \kstruct{snd_soc_dapm_widget} part of the registered
  \kstruct{snd_soc_card}.
  \begin{block}{\kfile{sound/soc/atmel/atmel_wm8904.c}}
    \fontsize{10}{10}\selectfont
    \begin{minted}{c}
static const struct snd_soc_dapm_widget atmel_asoc_wm8904_dapm_widgets[] = {
        SND_SOC_DAPM_HP("Headphone Jack", NULL),
        SND_SOC_DAPM_MIC("Mic", NULL),
        SND_SOC_DAPM_LINE("Line In Jack", NULL),
};
    \end{minted}
  \end{block}
\end{frame}

\begin{frame}[fragile]{Clocking: producer/consumer}
  The producer/consumer relationship is declared part of the
  \code{.dai_fmt} field of \kstruct{snd_soc_dai_link}.
  \begin{block}{\kfile{include/sound/soc.h}}
    \fontsize{6}{6}\selectfont
    \begin{minted}{c}
/*
 * DAI hardware clock providers/consumers
 *
 * This is wrt the codec, the inverse is true for the interface
 * i.e. if the codec is clk and FRM provider then the interface is
 * clk and frame consumer.
 */
#define SND_SOC_DAIFMT_CBP_CFP                (1 << 12) /* codec clk provider & frame provider */
#define SND_SOC_DAIFMT_CBC_CFP                (2 << 12) /* codec clk consumer & frame provider */
#define SND_SOC_DAIFMT_CBP_CFC                (3 << 12) /* codec clk provider & frame consumer */
#define SND_SOC_DAIFMT_CBC_CFC                (4 << 12) /* codec clk consumer & frame consumer */

/* previous definitions kept for backwards-compatibility, do not use in new contributions */
#define SND_SOC_DAIFMT_CBM_CFM                SND_SOC_DAIFMT_CBP_CFP
#define SND_SOC_DAIFMT_CBS_CFM                SND_SOC_DAIFMT_CBC_CFP
#define SND_SOC_DAIFMT_CBM_CFS                SND_SOC_DAIFMT_CBP_CFC
#define SND_SOC_DAIFMT_CBS_CFS                SND_SOC_DAIFMT_CBC_CFC
    \end{minted}
  \end{block}
  \begin{block}{\kfile{sound/soc/atmel/atmel_wm8904.c}}
    \fontsize{9}{9}\selectfont
    \begin{minted}{c}
          .dai_fmt = SND_SOC_DAIFMT_I2S
                | SND_SOC_DAIFMT_NB_NF
                | SND_SOC_DAIFMT_CBM_CFM,
    \end{minted}
  \end{block}
\end{frame}

\begin{frame}[fragile]{Clocking: dynamically changing clocks}
  The \code{.ops} member of \kstruct{snd_soc_dai_link} contains
  useful callbacks.
  \begin{block}{\kfile{include/sound/soc.h}}
    \fontsize{10}{10}\selectfont
    \begin{minted}{c}
/* SoC audio ops */
struct snd_soc_ops {
        int (*startup)(struct snd_pcm_substream *);
        void (*shutdown)(struct snd_pcm_substream *);
        int (*hw_params)(struct snd_pcm_substream *, struct snd_pcm_hw_params *);
        int (*hw_free)(struct snd_pcm_substream *);
        int (*prepare)(struct snd_pcm_substream *);
        int (*trigger)(struct snd_pcm_substream *, int);
};
    \end{minted}
  \end{block}
  \code{.hw_params} is called when setting up the audio stream. The
  \kstruct{snd_pcm_hw_params} contains the audio characteristics.
  Use \code{params_rate()} to get the sample rate,
  \code{params_channels} for the number of channels and
  \code{params_format} to get the format (including the bit depth).
  Finally, \code{snd_soc_params_to_bclk} calculates the bit clock.
\end{frame}

\begin{frame}[fragile]{Clocking: \code{hw_params}}
  \begin{itemize}
  \item \code{params_rate} gets the sample rate
  \item \code{params_channels} gets the number of channels
  \item \code{params_format} gets the format (including the bit depth)
  \item \code{snd_soc_params_to_bclk} calculates the bit clock.
  \item \code{snd_soc_dai_set_sysclk} sets the clock rate and
    direction for the DAI (SoC or codec)
    \begin{block}{}
      \begin{minted}{c}
int snd_soc_dai_set_sysclk(struct snd_soc_dai *dai, int clk_id,
        unsigned int freq, int dir);
      \end{minted}
    \end{block}
  \item it is also possible to configure the PLLs and clock divisors
    if necessary
    \begin{block}{}
      \fontsize{9}{9}\selectfont
      \begin{minted}{c}
int snd_soc_dai_set_clkdiv(struct snd_soc_dai *dai,
        int div_id, int div);
int snd_soc_dai_set_pll(struct snd_soc_dai *dai,
        int pll_id, int source, unsigned int freq_in, unsigned int freq_out);
      \end{minted}
    \end{block}
  \end{itemize}
\end{frame}

\begin{frame}[fragile]{Clocking example}
  \begin{block}{\kfile{sound/soc/atmel/atmel_wm8904.c}}
    \fontsize{9}{9}\selectfont
    \begin{minted}{c}
static int atmel_asoc_wm8904_hw_params(struct snd_pcm_substream *substream,
                struct snd_pcm_hw_params *params)
{
        struct snd_soc_pcm_runtime *rtd = substream->private_data;
        struct snd_soc_dai *codec_dai = rtd->codec_dai;
        int ret;

        ret = snd_soc_dai_set_pll(codec_dai, WM8904_FLL_MCLK, WM8904_FLL_MCLK,
                32768, params_rate(params) * 256);
        if (ret < 0) {
                pr_err("%s - failed to set wm8904 codec PLL.", __func__);
                return ret;
        }
    \end{minted}
  \end{block}
\end{frame}

\begin{frame}[fragile]{Clocking example}
  \begin{block}{\kfile{sound/soc/atmel/atmel_wm8904.c}}
    \fontsize{9}{9}\selectfont
    \begin{minted}{c}
        /*
         * As here wm8904 use FLL output as its system clock
         * so calling set_sysclk won't care freq parameter
         * then we pass 0
         */
        ret = snd_soc_dai_set_sysclk(codec_dai, WM8904_CLK_FLL,
                        0, SND_SOC_CLOCK_IN);
        if (ret < 0) {
                pr_err("%s -failed to set wm8904 SYSCLK\n", __func__);
                return ret;
        }

        return 0;
}

static struct snd_soc_ops atmel_asoc_wm8904_ops = {
        .hw_params = atmel_asoc_wm8904_hw_params,
};
    \end{minted}
  \end{block}
\end{frame}
