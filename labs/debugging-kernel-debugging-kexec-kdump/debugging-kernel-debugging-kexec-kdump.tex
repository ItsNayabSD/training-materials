\subchapter
{Kernel debugging: post-mortem analysis with kexec \& kdump}
{Objectives:
  \begin{itemize}
	  \item Setting up {\em Kexec \& kdump}.
	  \item Extracting a coredump for a crashed kernel.
  \end{itemize}
}

\section{kdump \& kexec}

As presented in the course, kdump/kexec allows to boot a new kernel and dump a
perfect copy of the crashed kernel (memory, registers, etc) which can be then
debugged using gdb or crash. 

\subsection{Building the dump-capture kernel}

We will now build the dump-capture kernel which will be booted on crash using
kexec. For that, we will use a simple buildroot image with a builtin initramfs
using the following commands on the development host:

\begin{bashinput}
$ cd /home/$USER/debugging-labs/buildroot
$ make O=build_kexec stm32mp157a_dk1_debugging_kexec_defconfig
$ make O=build_kexec
\end{bashinput}

Then, we'll copy the zImage and the device-tree to the nfs /root/kexec
directory:

\begin{bashinput}
$ mkdir /home/$USER/debugging-labs/nfsroot/root/kexec
$ cp build_kexec/images/zImage /home/$USER/debugging-labs/nfsroot/root/kexec
$ cp build_kexec/images/stm32mp157a-dk1.dtb /home/$USER/debugging-labs/nfsroot/root/kexec
\end{bashinput}

These files are now ready to be used from the target using kexec.

\subsection{Configuring kexec}

First of all we need to setup a kexec suitable memory zone for our crash kernel
image. This is achieved via the linux command line. Reboot, interrupt U-Boot and
add the \code{crashkernel=60M} parameter. This will tell the kernel to reserve
60M of memory to load a "rescue" kernel that will be booted on panic. We will
also add an option which will panic the kernel on oops to allow executing the
kexec kernel.

\begin{bashinput}
STM32MP> env edit bootargs
STM32MP> <existing bootargs> crashkernel=60M oops=panic
STM32MP> boot
\end{bashinput}

To load the crash kernel into the previously reserved memory zone, run the
following command on the target:

\begin{bashinput}
# kexec --type zImage -p /root/kexec/zImage --dtb=/root/kexec/stm32mp157a-dk1.dtb
  --command-line="console=ttySTM0,115200n8 maxcpus=1 reset_devices"
\end{bashinput}

Once done, you can trigger a crash using the previously mentioned watchdog
command:

\begin{bashinput}
$ watchdog -T 10 -t 5 /dev/watchdog0
\end{bashinput}

At this moment, the kernel will reboot into a new kernel using the specified
kernel after displaying the backtrace and a message:

\begin{bashinput}
[ 1181.987971] Loading crashdump kernel...
[ 1181.990839] Bye!
\end{bashinput}

After reboot, log into the new kernel normally and bring up the eth0 interface:
(We will use 192.168.0.101 to avoid cloberring ssh \code{know_hosts} file for
the 192.168.0.100 entry).
\begin{bashinput}
$ ifconfig eth0 192.168.0.101
\end{bashinput}

\textbf{Note:} ethernet setup might timeout due to some init issues after kexec
boot so this commands needs to be run another time.

\begin{bashinput}
$ cd /home/$USER/debugging-labs/
$ scp root@192.168.0.101:/proc/vmcore .
\end{bashinput}

You can load the \code{vmcore} file using cross GDB:

\begin{bashinput}
$ ${CROSS_COMPILE}gdb /home/$USER/debugging-labs/buildroot/output/build/linux-%\workingkernel%/vmlinux vmcore
\end{bashinput}
