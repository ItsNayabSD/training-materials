\subchapter{Accessing I/O memory and ports}{Objective: read /
  write data from / to a hardware device}

Throughout the upcoming labs, we will implement a character driver
allowing to write data to additional CPU serial ports available on
the BeaglePlay, and to read data from them.

After this lab, you will be able to:

\begin{itemize}
\item Add UART devices to the board device tree.
\item Access I/O registers to control the device and
      send first characters to it.
\end{itemize}

\section{Setup}

Go to your kernel source directory and continue working with the
\code{beagleplay-custom} branch, this way we can keep the same custom Device
Tree we already created, and build on top of it.

\section{Add UART devices}

In the following labs, we will be using UART5 and UART6, which are both
routed to the Mikrobus connector.

Before developing a driver for these additional UARTs on the board, we
need to find the corresponding Mikrobus pins.

First, open the Beagle Play hardware schematics and search for references to
\code{UART5_RX}, \code{UART5_TX}, \code{UART6_RX} and \code{UART6_TX}. If you
follow the UART5 signals, you'll see that they are already routed to the pins
labeled "TX" and "RX" on the Mikrobus connector. For UART6, you'll see that the
corresponding pins are used by the SPI2 bus by default, and that the pins for
\code{UART6_TX} and \code{UART6_RX} are respectively routed to the MOSI (COPI) and
MISO (CIPO) pins on the Mikrobus connector.

Go to the AM625 datasheet and find the pinmuxing settings for \code{UART6_TX} and \code{UART6_RX}.

The pinmuxing configuration is already done for UART5. For UART6, you'll have to add
a pinmuxing section using the information you found in the datasheet. You'll also have to disable the \code{main_spi2} by setting its \code{status} property to \code{"disabled"} in your custom device tree.

{\scriptsize \sourcecode{labs/kernel-beagle-serial-iomem/uarts-pinctrl.dts}}

Using a new USB-serial cable with male connectors, provided by your
instructor, connect your PC to UART5. The wire colors are the same
as for the cable that you're using for the console.

Then, declare the corresponding devices:

\sourcecode{labs/kernel-beagleplay-serial-iomem/uarts.dts}

This is a good example of how we can override definitions in the Device Tree.
\code{uart5} and \code{uart6} are already enabled and muxed in
\kfile{arch/arm64/boot/dts/ti/k3-am625-beagleplay.dts}. In the above code, we
just override the \code{compatible} property to use our driver instead of using
the default one.

Compile and update your DTB.

\section{Operate a platform device driver}

Go to the \code{~/linux-kernel-labs/modules/nfsroot/root/serial/} directory.
You will find a \code{serial.c} file which already provides a platform
driver skeleton.

Add the code needed to match the driver with the devices which you have
just declared in the device tree.

Compile your module and load it on your target. Check the kernel log
messages, that should confirm that the \code{probe()} routine was
called\footnote{Don't be surprised if the \code{probe()} routine is
actually called twice! That's because we have declared two devices.
Even if we only connect a serial-to-USB dongle to one of them, both
of them are ready to be used!}.

\section{Create a device private structure}

In the same way as in the nunchuk lab, we now need to create a
structure that will hold device specific information and help
keeping pointers between logical and physical devices.

As the first thing to store will be the base virtual address for
each device, let's declare this structure as follows:

\begin{verbatim}
struct serial_dev {
        void __iomem *regs;
};
\end{verbatim}

The first thing to do is allocate such a structure at the beginning
of the \code{probe()} routine. Let's do it with the \kfunc{devm_kzalloc}
function again as in the previous lab. Again, resource deallocation is
automatically taken care of when we use the \code{devm_} functions.

So, add the below line to your code:

\begin{verbatim}
struct serial_dev *serial;
...
serial = devm_kzalloc(&pdev->dev, sizeof(*serial), GFP_KERNEL);
if (!serial)
        return -ENOMEM;
\end{verbatim}

\section{Get a base virtual address for your device registers}

You can now get a virtual address for your device's base physical
address, by calling:

\begin{verbatim}
serial->regs = devm_platform_ioremap_resource(pdev, 0);
if (IS_ERR(serial->regs))
        return PTR_ERR(serial->regs);
\end{verbatim}

What's nice is that you won't ever have to release this resource,
neither in the \code{remove()} routine, nor if there are failures
in subsequent steps of the \code{probe()} routine.

Make sure that your updated driver compiles, loads and unloads well.

\section{Device initialization}

Now that we have a virtual address to access registers, we are ready to
configure a few registers which will allow us to enable the UART
devices. Of course, this will be done in the \code{probe()} routine.

\subsection{Accessing device registers}

As we will have multiple registers to read, create a \code{reg_read()}
routine, returning a \code{u32} value, and taking a \code{serial}
pointer to a \code{serial_dev} structure and an \code{unsigned int}
register offset.

Your prototype should look like:
\begin{verbatim}
static u32 reg_read(struct serial_dev *serial, unsigned int reg);
\end{verbatim}

In this function, read from a 32 bits register at the base virtual
address for the device, plus the register offset multiplied by 4.

All the UART register offsets have standardized values, shared between
several types of serial drivers (see
\kfile{include/uapi/linux/serial_reg.h}). This explains why they are not
completely ready to use and we have to multiply them by 4 for OMAP SoCs.

Create a similar \code{reg_write()} routine, writing an \code{int}
value at a given register offset (don't forget to multiply it by 4) from
the device base virtual address. The following code samples are using
the \kfunc{writel} convention of passing the value first, then the
offset. Your prototype should look like:
\begin{verbatim}
static void reg_write(struct serial_dev *serial, u32 val, unsigned int reg);
\end{verbatim}

In the next sections, we will tell you what register offsets to use
to drive the hardware.

\subsection{Power management initialization}

Add the below lines to the probe function:

\begin{verbatim}
pm_runtime_enable(&pdev->dev);
pm_runtime_get_sync(&pdev->dev);
\end{verbatim}

And add the below line to the \code{remove()} routine:

\begin{verbatim}
pm_runtime_disable(&pdev->dev);
\end{verbatim}

\subsection{Line and baud rate configuration}

After these lines, let's add code to initialize the line
and configure the baud rate. This shows how to get a special
property from the device tree, in this case \code{clock-frequency}:

\sourcecode{labs/kernel-beagleplay-serial-iomem/uart-line-init.c}

Declare \code{baud_divisor} and \code{uartclk} as \code{unsigned int}.

\subsection{FIFOs reset}

The last thing to do is to reset the FIFOs:

\sourcecode{labs/kernel-beagleplay-serial-iomem/uart-line-reset.c}

We are now ready to transmit characters over the serial ports!

If you have a bit of spare time, you can look at section 12.2.4 of the
AM62x TRM for details about how to use the UART ports, to understand
better what we are doing here.

\section{Standalone write routine}

Implement a C routine taking a pointer to a \code{serial_dev}
structure and one character as parameters, and writing
this character to the serial port, using the following steps:

\begin{enumerate}
\item Wait until the \ksym{UART_LSR_THRE} bit gets set in the
  \ksym{UART_LSR} register.  You can busy-wait for this condition to happen.
  In the busy-wait loop, you can call the \kfunc{cpu_relax} kernel function
  to ensure the compiler won't optimise away this loop.
\item Write the character to the \ksym{UART_TX} register.
\end{enumerate}

Add a call to this routine from your module \code{probe()} function,
and recompile your module.

Open a new \code{picocom} instance on your new serial port (not the
serial console):

\begin{verbatim}
picocom -b 115200 /dev/ttyUSB1
\end{verbatim}

Load your module on the target. You should see the
corresponding character in the new \code{picocom} instance,
showing what was written to UART2.

You can also check that you also get the same character on UART4
(just connect to the UART4 pins instead of the UART2 ones).

\section{Driver sanity check}

Remove your module and try to load it again. If the second attempt to
load the module fails, it is probably because your driver doesn't
properly free the resources it allocated or registered, either at module
exit time, or after a failure during the module \code{probe()} function. Check
and fix your module code if you have such problems.
