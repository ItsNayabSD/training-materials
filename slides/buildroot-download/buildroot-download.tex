\setbeamerfont{block title}{size=\scriptsize}

\section{Download infrastructure in Buildroot}

\begin{frame}{Introduction}
  \begin{itemize}
  \item One important aspect of Buildroot is to fetch source code or
    binary files from third party projects.
  \item Download supported from HTTP(S), FTP, Git, Subversion, CVS,
    Mercurial, etc.
  \item Being able to do reproducible builds over a long period of
    time requires understanding the download infrastructure.
  \end{itemize}
\end{frame}

\begin{frame}{Download location}
  \begin{itemize}
  \item Each Buildroot package indicates in its \code{.mk} file which
    files it needs to be downloaded.
  \item Can be a tarball, one or several patches, binary files, etc.
  \item When downloading a file, Buildroot will successively try the
    following locations:
    \begin{enumerate}
    \item The local \code{$(DL_DIR)} directory where downloaded files
      are kept
    \item The {\bf primary site}, as indicated by \code{BR2_PRIMARY_SITE}
    \item The {\bf original site}, as indicated by the package
      \code{.mk} file
    \item The {\bf backup Buildroot mirror}, as indicated by
      \code{BR2_BACKUP_SITE}
    \end{enumerate}
  \end{itemize}
\end{frame}

\begin{frame}{{\tt DL\_DIR}}
  \begin{itemize}
  \item Once a file has been downloaded by Buildroot, it is cached in
    the directory pointed by
    \code{$(DL_DIR)}, in a sub-directory named after the package.
  \item By default, \code{$(TOPDIR)/dl}
  \item Can be changed
    \begin{itemize}
    \item using the \code{BR2_DL_DIR} configuration option
    \item or by passing the \code{BR2_DL_DIR} environment variable,
      which overrides the config option of the same name
    \end{itemize}
  \item The download mechanism is written in a way that allows
    independent parallel builds to share the same \code{DL_DIR} (using
    atomic renaming of files)
  \item No cleanup mechanism: files are only added, never removed,
    even when the package version is updated.
  \end{itemize}
\end{frame}

\begin{frame}{Primary site}
  \begin{itemize}
  \item The \code{BR2_PRIMARY_SITE} option allows to define the
    location of a HTTP or FTP server.
  \item By default empty, so this feature is disabled.
  \item When defined, used in priority over the original location.
  \item Allows to do a local mirror, in your company, of all the files
    that Buildroot needs to download.
  \item When option \code{BR2_PRIMARY_SITE_ONLY} is enabled, only the
    {\em primary site} is used
    \begin{itemize}
    \item It does not fall back on the original site and the backup
      Buildroot mirror
    \item Guarantees that all downloads must be in the primary site
    \end{itemize}
  \end{itemize}
\end{frame}

\begin{frame}{Backup Buildroot mirror}
  \begin{itemize}
  \item Since sometimes the upstream locations disappear or are
    temporarily unavailable, having a backup server is useful
  \item Address configured through \code{BR2_BACKUP_SITE}
  \item Defaults to \url{http://sources.buildroot.net}
    \begin{itemize}
    \item maintained by the Buildroot community
    \item updated before every Buildroot release to contain the
      downloaded files for all packages
    \item exception: cannot store all possible versions for packages
      that have their version as a configuration option. Generally
      only affects the kernel or bootloader, which typically don't
      disappear upstream.
    \end{itemize}
  \end{itemize}
\end{frame}

\begin{frame}{Special case of VCS download}
  \begin{itemize}
  \item When a package uses the source code from Git, Subversion or
    another VCS, Buildroot cannot directly download a tarball.
  \item It uses a VCS-specific method to fetch the specified version
    of the source from the VCS repository
  \item The source code is checked-out/cloned inside \code{DL_DIR} and
    kept to speed-up further downloads of the same project (caching
    only implemented for Git)
  \item Finally a tarball containing only the source code (and not the
    version control history or metadata) is created and stored in
    \code{DL_DIR}
    \begin{itemize}
    \item Example: \code{avrdude-eabe067c4527bc2eedc5db9288ef5cf1818ec720.tar.gz}
    \end{itemize}
  \item This tarball will be re-used for the next builds, and attempts
    are made to download it from the primary and backup sites.
  \item Due to this, always use a tag name or a full commit id, and
    never a branch name: the code will never be re-downloaded when the
    branch is updated.
  \end{itemize}
\end{frame}

\begin{frame}{Vendoring}
  \begin{itemize}
  \item Some language-specific package management systems like to
    download the dependencies by themselves: {\em vendoring}
  \item Examples: {\em Cargo} in the Rust ecosystem, or {\em Go}
  \item Problem for build systems: reproducibility of the builds,
    licensing, offline builds
  \item Buildroot supports vendoring dependencies for {\em cargo} and
    {\em go} packages
  \item Right after the download of the package source code, Buildroot
    invokes the language-specific vendoring tool, and bundles the
    dependencies inside the tarball
  \end{itemize}
\end{frame}

\begin{frame}[fragile]{File integrity checking}
  \begin{itemize}
  \item Buildroot packages can provide a \code{.hash} file to provide
    {\em hashes} for the downloaded files.
  \item The download infrastructure uses this hash file when available
    to check the integrity of the downloaded files.
  \item Hashes are checked every time a downloaded file is used, even
    if it is already cached in \code{$(DL_DIR)}.
  \item If the hash is incorrect, the download infrastructure attempts
    to re-download the file once. If that still fails, the build
    aborts with an error.
  \end{itemize}

  \begin{block}{Hash checking message}
{\scriptsize
\begin{verbatim}
strace-4.10.tar.xz: OK (md5: 107a5be455493861189e9b57a3a51912)
strace-4.10.tar.xz: OK (sha1: 5c3ec4c5a9eeb440d7ec70514923c2e7e7f9ab6c)
>>> strace 4.10 Extracting
\end{verbatim}}
  \end{block}
\end{frame}

\begin{frame}{Download-related {\tt make} targets}
  \begin{itemize}
  \item \code{make source}
    \begin{itemize}
    \item Triggers the download of all the files needed to build the
      current configuration.
    \item All files are stored in \code{$(DL_DIR)}
    \item Allows to prepare a fully offline build
    \end{itemize}
  \item \code{make external-deps}
    \begin{itemize}
    \item Lists the files from \code{$(DL_DIR)} that are needed for
      the current configuration to build.
    \item Does not guarantee that all files are in \code{$(DL_DIR)}, a
      \code{make source} is required
    \end{itemize}
  \end{itemize}
\end{frame}
