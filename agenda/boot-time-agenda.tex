\documentclass[a4paper,12pt,obeyspaces,spaces,hyphens]{article}

\def \trainingtype{onsite}
\def \agendalanguage{english}

\usepackage{ifthen}
\def \training{boot-time}

% Title
\ifthenelse{\equal{\agendalanguage}{french}}{
  \def \trainingtitle{Formation optimisation du temps de démarrage de Linux embarqué}
}{
  \def \trainingtitle{Embedded Linux boot time optimization training}
}

% Duration
\ifthenelse{\equal{\trainingtype}{online}}{
  \def \trainingduration{4}
}{
  \def \trainingduration{3}
}

% Training objectives
\ifthenelse{\equal{\agendalanguage}{french}}{
  \def \traininggoals{
    \begin{itemize}
    \item Être capable d'utiliser les outils et techniques pour mesurer
      le temps de démarrage d'un système embarqué.
    \item Être capable de réduire le temps de démarrage au niveau de
      l'initialisation de l'espace utilisateur Linux.
    \item Être capable de réduire le temps de démarrage au niveau de
      l'initialisation du noyau Linux.
    \item Être capable de réduire le temps de démarrage au niveau de
      l'initialisation du chargeur de démarrage.
    \item Être capable de mettre en oeuvre d'autres techniques avancées
      et alternatives d'optimisation du temps de démarrage.
    \end{itemize}
  }
}{
  \def \traininggoals{
    \begin{itemize}
    \item Be able to use various tools and techniques to measure the
      boot time of an embedded Linux system.
    \item Be able to reduce the boot time spent during the {\em
        user-space} initialization.
    \item Be able to reduce the boot time spent during the {\em
        kernel} initialization.
    \item Be able to reduce the boot time spent during the {\em
        bootloader} initialization.
    \item Be able to use advanced and alternatives techniques of boot
      time optimization.
    \end{itemize}
  }
}

% Training prerequisites
\def \trainingprerequisites{
  \begin{itemize}
    \prerequisitecommandline
    \prerequisiteembeddedlinux
    \prerequisiteenglish
  \end{itemize}
}

% Training audience
\ifthenelse{\equal{\agendalanguage}{french}}{
  \def \trainingaudience{
    Sociétés et ingénieurs développeurs de systèmes Linux embarqués.
    \newline Personnes offrant de l'assistance à de tels développeurs.
  }
}{
  \def \trainingaudience{
    People developing embedded Linux systems.
    \newline People supporting embedded Linux system developers.
  }
}

% Time ratio
\def \onsitelecturetimeratio{40}
\def \onsitelabtimeratio{60}


\usepackage{agenda_old}

\begin{document}

\feshowtitle

\feshowinfo

\feagendatwocolumn
{Hardware}
{
  The hardware platform used for the practical labs of this training
  session is the {\bf BeagleBone Black} board, which features:

  \begin{itemize}
  \item An ARM AM335x processor from Texas Instruments (Cortex-A8
    based), 3D acceleration, etc.
  \item 512 MB of RAM
  \item 2 GB of on-board eMMC storage
        \newline(4 GB in Rev C)
  \item USB host and device
  \item HDMI output
  \item 2 x 46 pins headers, to access UARTs, SPI buses, I2C buses
    and more.
  \end{itemize}
}
{}
{
  \begin{center}
    \includegraphics[height=5cm]{../slides/beagleboneblack-board/beagleboneblack.png}
  \end{center}
}

\feagendaonecolumn
{Practical labs}
{
  The practical labs of this training session use the following
  hardware peripherals:

  \begin{itemize}
  \item A USB webcam
  \item An LCD and touchscreen cape connected to the
    BeagleBone Black board, to display the video captured by the webcam.
  \end{itemize}
}

\section{Day 1 - Morning}

\feagendatwocolumn
{Lecture - Principles}
{
  \begin{itemize}
  \item How to measure boot time
  \item Main ideas
  \end{itemize}
}
{Lab - Preparing the system}
{
 \begin{itemize}
 \item Downloading bootloader, kernel and Buildroot source code
 \item Board setup, setting up serial communication
 \item Configure Buildroot and build the system
 \item Configure and build the U-Boot bootloader. Prepare an SD card
       and boot the bootloader from it.
 \item Configure and build the kernel. Boot the system
 \end{itemize}
}

\section{Day 1 - Afternoon}

\feagendatwocolumn
{Lecture - Measuring time}
{
  \begin{itemize}
  \item Generic software techniques
  \item Hardware techniques
  \item Specific solutions for each stage
  \end{itemize}
}
{Lab - Measuring time - Software solution}
{
 \begin{itemize}
 \item Modify the system to measure time at various steps
 \item Timing messages on the serial console
 \item Timing the execution of the application
 \end{itemize}
}

\feagendatwocolumn
{Lecture - Toolchain optimizations}
{
  \begin{itemize}
  \item Introduction to toolchains
  \item C libraries
  \item Size information
  \item Measuring executable performance with \code{time}
  \end{itemize}
}
{Lab - Toolchain optimizations}
{
  \begin{itemize}
  \item Measuring application execution time
  \item Switching to a Thumb2 toolchain
  \item Generate a Buildroot SDK to rebuild faster
  \end{itemize}
}

\section{Day 2- Morning}

\feagendatwocolumn
{Lecture - Application optimization}
{
  \begin{itemize}
  \item Using \code{strace} and \code{ltrace}
  \item Other profiling techniques
  \end{itemize}
}
{Lab - Application optimization}
{
 \begin{itemize}
 \item Finding unnecessary configuration options in applications
 \item Modifying configuration options through Buildroot
 \item Experiments with \code{strace} to trace program execution
 \end{itemize}
}

\feagendatwocolumn
{Lecture - Optimizing system initialization}
{
  \begin{itemize}
  \item Using BusyBox \code{bootchartd}
  \item Optimizing init scripts
  \item Possibility to start your application directly
  \end{itemize}
}
{Lab - Optimizing system initialization}
{
 \begin{itemize}
 \item Using Buildroot to remove unnecessary scripts and commands
 \item Access-time based technique to identify  unused files
 \item Simplifying BusyBox
 \item Starting the application as the init program
 \end{itemize}
}

\section{Day 2 - Afternoon}

\feagendatwocolumn
{Lecture - Filesystem optimizations}
{
  \begin{itemize}
  \item Available filesystems, performance and boot time aspects
  \item Making UBIFS faster
  \item Tweaks for reducing boot time
  \item Booting on an initramfs
  \item Using static executables: licensing constraints
  \end{itemize}
}
{Lab - Filesystem optimizations}
{
 \begin{itemize}
 \item Trying and measuring two block filesystems: ext4 and SquashFS.
 \item Trying and measuring the initramfs solution. Constraints
       due to this solution.
 \end{itemize}
}

\feagendatwocolumn
{Lecture - Kernel optimizations}
{
  \begin{itemize}
  \item Using {\em Initcall debug} to generate a boot graph
  \item Compression and size features
  \item Reducing or suppressing console output
  \item Multiple tweaks to reduce boot time
  \end{itemize}
}
{Lab - Kernel optimizations}
{
 \begin{itemize}
 \item Generating and analyzing a boot graph for the kernel
 \item Find and eliminate unnecessary kernel features
 \item Find the best kernel compression solution for our system
 \end{itemize}
}

\section{Day 3 - Morning}

\feagendaonecolumn
{Lab - Kernel optimizations}
{
 \begin{itemize}
 \item Continued from Day 2
 \end{itemize}
}

\section{Day 3 - Afternoon}

\feagendatwocolumn
{Lecture - Bootloader optimizations}
{
  \begin{itemize}
  \item Generic tips for reducing U-Boot's size and boot time
  \item Optimizing U-Boot scripts and kernel loading
  \item Skipping the bootloader - How to modify U-Boot to
        enable its {\em Falcon mode}
  \end{itemize}
}
{Lecture - U-Boot Falcon mode}
{
  \begin{itemize}
  \item Principles and goals
  \item The Device Tree preparation work that U-Boot does to prepare Linux kernel booting
  \item Using the \code{spl export} command to do this work in advance
  \item Modifying U-Boot's source code and configuring it for directly
        booting Linux and skipping the U-Boot second stage.
  \item Example instructions and setups for booting from MMC and NAND flash
  \item How to debug Falcon mode
  \item How to fall back to U-Boot
  \item Limitations
  \end{itemize}
}

\feagendaonecolumn
{Lab - Bootloader optimizations}
{
 \begin{itemize}
 \item Using the above techniques to make the bootloader
       as quick as possible.
 \item Switching to faster storage
 \item Configuring U-Boot for {\em Falcon mode} booting,
       skipping U-Boot's second stage.
 \end{itemize}
}

\feagendaonecolumn
{Wrap-up - Achieved results}
{
 \begin{itemize}
 \item Sharing and comparing results achieved by the various groups
 \item Questions and answers, experience sharing with the trainer
 \end{itemize}
}

\end{document}
