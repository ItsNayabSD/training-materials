\section{Application Tracing}

\subsection{strace}

\input{../common/strace.tex}

\subsection{ltrace}

\input{../common/ltrace.tex}

\subsection{LD\_PRELOAD}

\begin{frame}
  \frametitle{Shared libraries}
  \begin{itemize}
    \item Shared libraries are provided as {\em .so} files that are actually ELF files
    \begin{itemize}
      \item Loaded at startup by \code{ld.so} (the dynamic loader)
      \item Or at runtime using \code{dlopen()} from your code
    \end{itemize}
    \item When starting a program (an ELF file actually), the kernel will
          parse it and load the interpreter that needs to be invoked.
    \begin{itemize}
      \item Most of the time \code{PT_INTERP} program header of the ELF file is
            set to \code{ld-linux.so}.
    \end{itemize}
    \item At loading time, the dynamic loader \code{ld.so} will resolve all the
          symbols that are present in dynamic libraries.
    \item Shared libraries are loaded only once by the OS and then mappings are
          created for each application that uses the library.
    \begin{itemize}
      \item This allows to reduce the memory used by libraries.
    \end{itemize}
  \end{itemize}
\end{frame}

\begin{frame}
  \frametitle{Hooking Library Calls}
  \begin{itemize}
    \item In order to do some more complex library call hooks, one can use
          the {\em LD\_PRELOAD} environment variable.
    \item {\em LD\_PRELOAD} is used to specify a shared library that will be
          loaded before any other library by the dynamic loader.
    \item Allows to intercept all library calls by preloading another library.
    \begin{itemize}
      \item Overrides libraries symbols that have the same name.
      \item Allows to redefine only a few specific symbols.
      \item "Real" symbol can still be loaded and used with \code{dlsym} (\manpage{dlsym}{3})
    \end{itemize}
    \item Used by some debugging/tracing libraries ({\em libsegfault},
          {\em libefence})
    \item Works for C and C++.
  \end{itemize}
\end{frame}

\begin{frame}[fragile]
  \frametitle{{\em LD\_PRELOAD} example 1/2}
  \begin{itemize}
    \item Library snippet that we want to preload using {\em LD\_PRELOAD}:
  \end{itemize}
  \begin{block}{}
    \begin{minted}[fontsize=\small]{c}
#include <string.h>
#include <unistd.h>

ssize_t read(int fd, void *data, size_t size) {
  memset(data, 0x42, size);
  return size;
}
    \end{minted}
  \end{block}
  \begin{itemize}
    \item Compilation of the library for {\em LD\_PRELOAD} usage:
  \end{itemize}
  \begin{block}{}
    \begin{minted}[fontsize=\small]{console}
$ gcc -shared -fPIC -o my_lib.so my_lib.c
    \end{minted}
  \end{block}

  \begin{itemize}
    \item Preloading the new library using {\em LD\_PRELOAD}:
  \end{itemize}
  \begin{block}{}
    \begin{minted}[fontsize=\small]{console}
$ LD_PRELOAD=./my_lib.so ./exe
    \end{minted}
  \end{block}
\end{frame}

\begin{frame}[fragile]
  \frametitle{{\em LD\_PRELOAD} example 2/2}
  \begin{itemize}
    \item Chaining a call to the real symbol to avoid altering the
	    application behavior:
  \end{itemize}
  \begin{block}{}
    \begin{minted}[fontsize=\tiny]{c}
#include <stdio.h>
#include <unistd.h>
#include <dlfcn.h>

ssize_t read(int fd, void *data, size_t size)
{
        size_t (*read_func)(int, void *, size_t);
        void *handle;
        char *error;

        handle = dlopen("/lib/libc.so.6", RTLD_LAZY);
        if (!handle) {
                fprintf(stderr, "Can not find overriden library\n");
                return 0;
        }
        dlerror();
        read_func = dlsym(handle, "read");
        error = dlerror();
        if (error) {
                fprintf(stderr, "Can not find overriden symbol: %s\n", error);
                return 0;
        }
        fprintf(stderr, "Trying to read %lu bytes to %p from file descriptor %d\n", size, data, fd);
        return read_func(fd, data, size);
}

    \end{minted}
  \end{block}
\end{frame}

\subsection{uprobes and perf}

\begin{frame}{Probes in linux}
	\begin{itemize}
    \item The linux kernel is able to dynamically add some instrumentation
      (or "\textbf{probes}") to almost any code running on a platform,
      either in userspace, kernel space, or both.
    \item This mechanism works by "patching" the code at runtime to insert
      the probe. When the patched code is executed, the probe records the
      execution. It can also collect additional data.
    \item There are different kinds of probes exposed by the kernel:
			\begin{itemize}
        \item \textbf{uprobes}: hook on almost any userspace instruction
          and capture local data
        \item \textbf{uretprobes}: hook on userspace function exit and
          capture return value
        \item \textbf{entry fprobe}: hook on kernel function entry
        \item \textbf{exit fprobe}: hook on kernel function exit
        \item \textbf{kprobes}: hook on almost any kernel instruction and
          capture local data
        \item \textbf{kretprobe}: hook on kernel function exit and capture
          return value
			\end{itemize}
	\end{itemize}
\end{frame}

\begin{frame}[fragile]
  \frametitle{uprobes}
  \begin{itemize}
    \item {\em uprobe} is a probe mechanism offered by the kernel allowing
	    to trace userspace code.
    \item Can target any userspace instruction
    \begin{itemize}
      \item Internally patches the loaded \code{.text} section with breakpoints
        that are handled by the kernel trace system
    \end{itemize}
    \item Exposed by file \code{/sys/kernel/tracing/uprobe_events}
    \item User is expected to compute the offset of the targeted
	    instruction inside the corresponding VMA (containing the
	    \code{.text} section) of the targeted process
  \end{itemize}
  \begin{block}{}
	  \begin{minted}{bash}
echo 'p /bin/bash:0x4245c0' > /sys/kernel/tracing/uprobe_events
	  \end{minted}
  \end{block}
  \begin{itemize}
	  \item Uprobes are wrapped by some common tools (e.g:
		  \code{perf}, \code{bcc}) for easier usage
	  \item \kdochtml{trace/uprobetracer}
  \end{itemize}
\end{frame}

\begin{frame}[fragile]
  \frametitle{The {\em perf} tool}
  \begin{itemize}
    \item {\em perf} tool was started as a tool to profile application under
          Linux using performance counters (\manpage{perf}{1}).
    \item It became much more than that and now allows to manage tracepoints,
          kprobes and uprobes.
    \item {\em perf} can profile both user-space and kernel-space execution.
    \item {\em perf} is based on the \code{perf_event} interface that is
          exposed by the kernel.
    \item Provides a set of operations, each having specific arguments (see
          {\em perf} help).
    \begin{itemize}
      \item \code{stat}, \code{record}, \code{report}, \code{top}, \code{annotate}, \code{ftrace}, \code{list}, \code{probe}, etc
    \end{itemize}
  \end{itemize}
\end{frame}

\begin{frame}[fragile]
  \frametitle{Using {\em perf record}}
  \begin{itemize}
    \item {\em perf record} allows to record performance events per-thread,
          per-process and per-cpu basis.
    \item Kernel needs to be configured with \kconfigval{CONFIG_PERF_EVENTS}{y}.
    \item This is the first command that needs to be run to gather data from
          program execution and output them into \code{perf.data}.
    \item \code{perf.data} file can then be analyzed using \code{perf annotate}
          and \code{perf report}.
    \begin{itemize}
      \item Useful on embedded systems to analyze data on another computer.
    \end{itemize}
  \end{itemize}
\end{frame}

\begin{frame}[fragile]
  \frametitle{Probing userspace functions}
  \begin{itemize}
    \item List functions that can be probed in a specific
          executable:
  \begin{block}{}
    \begin{minted}[fontsize=\scriptsize]{C}
$ perf probe --source=<source_dir> -x my_app -F
    \end{minted}
  \end{block}
    \item List lines number that can be probed in a specific
          executable/function:
  \begin{block}{}
    \begin{minted}[fontsize=\scriptsize]{C}
$ perf probe --source=<source_dir> -x my_app -L my_func
    \end{minted}
  \end{block}
    \item Create uprobes on user-space library/executable functions:
  \begin{block}{}
    \begin{minted}[fontsize=\scriptsize]{C}
$ perf probe -x /lib/libc.so.6 printf
$ perf probe -x app my_func:3 my_var
$ perf probe -x app my_func%return ret=%r0
    \end{minted}
  \end{block}
  \item Record the execution of these tracepoints:
  \begin{block}{}
    \begin{minted}[fontsize=\scriptsize]{C}
$ perf record -e probe_app:my_func -e probe_libc:printf
    \end{minted}
  \end{block}
  \end{itemize}
\end{frame}

\setuplabframe
{Application tracing}
{
  Analyzing of application interactions
  \begin{itemize}
    \item Analyze dynamic library calls from an application using
            {\em ltrace}.
    \item Overriding a library function with \code{LD_PRELOAD}.
    \item Using {\em strace} to analyze program syscalls.
  \end{itemize}
}
