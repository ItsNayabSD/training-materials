\subchapter{Bootloader - TF-A and U-Boot}{Objectives: Set up serial
  communication, compile and install the U-Boot bootloader, use basic
  U-Boot commands, set up TFTP communication with the development
  workstation.}

As the bootloader is the first piece of software executed by a
hardware platform, the installation procedure of the bootloader is
very specific to the hardware platform. There are usually two cases:

\begin{itemize}

\item The processor offers nothing to ease the installation of the
  bootloader, in which case the JTAG has to be used to initialize
  flash storage and write the bootloader code to flash. Detailed
  knowledge of the hardware is of course required to perform these
  operations.

\item The processor offers a monitor, implemented in ROM, and through
  which access to the memories is made easier.

\end{itemize}

The STM32MP1 SoC, falls into the second category. The monitor
integrated in the ROM reads the SD card to search for a valid
bootloader (the boot mode is actually configurable via a few input
pins). In case no bootloader is found, it will operate in a fallback
mode, that will allow to use an external tool to reflash some
executable through USB. Therefore, either by using an MMC/SD card or
that fallback mode, we can start up an STM32MP1-based board without
having anything installed on it.
\section{Setup}

Go to the \code{$HOME/__SESSION_NAME__-labs/bootloader} directory.

\section{Setting up serial communication with the board}

Plug the USB-A to micro USB-B cable on the Discovery board. There is
only one micro USB port on the board, it is CN11, also named ST-LINK.
This is a debug interface and exposes multiple debugging interfaces,
including a serial interface. When plugged in your computer, a serial
port should appear, {\tt \hosttty}.

You can also see this device appear by looking at the output of
\code{sudo dmesg}.

To communicate with the board through the serial port, install a
serial communication program, such as \code{picocom}:

\bashcmd{$ sudo apt install picocom}

If you run {\tt ls -l \hosttty}, you can also see that only
\code{root} and users belonging to the \code{dialout} group have
read and write access to the serial console. Therefore, you need
to add your user to the \code{dialout} group:

\bashcmd{$ sudo adduser $USER dialout}

{\bf Important}: for the group change to be effective, you have to
reboot your computer (at least on Ubuntu 24.04) and log in again.
A workaround is to run \code{newgrp dialout}, but it is not global.
You have to run it in each terminal.

Run {\tt picocom -b 115200 \hosttty}, to start serial
communication on {\tt \hosttty}, with a baudrate of 115200.
If you wish to exit \code{picocom}, press \code{[Ctrl][a]} followed by
\code{[Ctrl][x]}.

Don't be surprised if you don't get anything on the serial console yet,
even if you reset the board. That's because the SoC has nothing to boot
on yet. We will prepare a micro SD card to boot on in the next paragraphs.

\section{TF-A and U-Boot relationship}

The boot process is done in two steps with the ROM monitor trying to
execute a first piece of software, called {\em fsbl}, from its
internal SRAM, that will initialize the DRAM, and a second program,
{\em ssbl} that will in turn load Linux and execute it.

In our case, {\em fsbl} is provided by TF-A BL2 and {\em
  ssbl} is provided by U-Boot.

TF-A BL2 is loading U-Boot from the Firmware Image Package ({\em
  FIP}), that will also contain the configuration for this second
part. The {\em FIP} is generated from TF-A sources, so first we are
going to build U-Boot.

\section{U-Boot setup}

Download U-Boot:

\begin{bashinput}
$ git clone https://gitlab.denx.de/u-boot/u-boot
$ cd u-boot
$ git checkout v2024.04
\end{bashinput}

Get an understanding of U-Boot's configuration and compilation steps
by reading the \code{README} file, and specifically the {\em Building
the Software} section.

Basically, you need to:

\begin{enumerate}

\item Specify the cross-compiler prefix
(the part before \code{gcc} in the cross-compiler executable name):
\bashcmd{$ export CROSS_COMPILE=arm-linux-}

\item Run \inlinebash{$ make <NAME>_defconfig}, where the list of available
  configurations can be found in the \code{configs/} directory. There
  are multiple stm32mp15 configurations. We will use the standard one
  (\code{stm32mp15}).

\item Now that you have a valid initial configuration, you can now
  run \inlinebash{$ make menuconfig} to further edit your bootloader features.

  \begin{itemize}

  \item In the \code{Environment} submenu, we will configure U-Boot so
    that it stores its environment inside a file called \code{uboot.env}
    in an ext4 filesystem:
    \begin{itemize}
    \item Disable \code{Environment is not stored}. We want changes to variables to
        be persistent across reboots
    \item Enable \code{Environment is in a EXT4 filesystem}. Disable all other
        options for environment storage (e.g. MMC, SPI, UBI)
% Environment in MMC also works, but we need to be careful with the size available
% and the offset defined. tests showed that if MMC is selected it will take
% precedence over ext4
    \item \code{Name of the block device for the environment}: \code{mmc}
    \item \code{Device and partition for where to store the environment in
        EXT4}: \code{0:4}
    \item \code{Name of the EXT4 file to use for the environment}: \code{/uboot.env}
    \end{itemize}

  \item In the \code{Device Drivers} $\rightarrow$ \code{Watchdog
      Timer Support} submenu, disable \code{IWDG watchdog
      driver for STM32 MP's family}, so that U-Boot doesn't start the
    watchdog.
  \end{itemize}

Install the following packages which should be needed to compile U-Boot for
your board:

\begin{bashinput}
$ sudo apt install libssl-dev device-tree-compiler swig \
       python3-dev python3-setuptools
\end{bashinput}

\item Finally, run \bashcmd{make DEVICE_TREE=stm32mp157a-dk1}
  which will build U-Boot
  \footnote{You can speed up the
  compiling by using the \code{-jX} option with \code{make}, where X
  is the number of parallel jobs used for compiling. Twice the
  number of CPU cores is a good value.}.
  The \code{DEVICE_TREE} variable specifies the specific
  Device Tree that describes our hardware board.
  You can see that in this case, U-Boot
  only ships a Device Tree for the board with the previous version
  of the chip ({\em stm32mp157a} instead of {\em stm32mp157d}).
  Alternatively, if you wish to run just \code{make},
  specify our board's device tree name on
  \code{Device Tree Control} $\rightarrow$ \code{Default Device Tree for DT Control}
  option.
\end{enumerate}

\section{TF-A setup}

Get the mainline TF-A sources:

\begin{bashinput}
$ cd ..
$ git clone https://git.trustedfirmware.org/TF-A/trusted-firmware-a.git
$ cd trusted-firmware-a/
$ git checkout v2.9.0
\end{bashinput}

Several configuration parameters have to be passed to the Makefile:
\begin{itemize}
\item Specify the cross-compiler prefix (the part before gcc in the
      cross-compiler executable name), either using the environment
      variable:\inlinebash{$ export CROSS_COMPILE=arm-linux-}, or just by
      adding it to the \code{make} commande line.
\item The architecture has to be selected: \code{ARCH=aarch32}, as
      well as the major version of Arm Architecture, here the Cortex A7 is
      an Armv7, so we need to use \code{ARM_ARCH_MAJOR=7}
\item The STM32MP1 platform is selected too with \code{PLAT=stm32mp1}
\item Specify the AArch32 Secure Payload component, we are going to
      use a minimal monitor implementation provided by TF-A: the
      {\em SP-MIN}. For this we need to add the following variable:
      \code{AARCH32_SP=sp_min}
\item For this specific board, the device tree is generated and then
      needs to be specifed: \code{DTB_FILE_NAME=stm32mp157a-dk1.dtb}
\item Specify the configuration of this firmware which is actually
      the Device Tree passed to U-Boot:
      \code{BL33_CFG=../u-boot/u-boot.dtb}
\item Specify that the fsbl will be located on the SD
      card with \code{STM32MP_SDMMC=1}.
\item Specify the location of the BL33 (Boot loader stage 3-3):
      \code{BL33=../u-boot/u-boot-nodtb.bin}
\end{itemize}

We can now generate the \code{bl32}, \code{dtb}, and \code{fip} targets
with a single command line:
\begin{bashinput}
$ make ARM_ARCH_MAJOR=7 ARCH=aarch32 PLAT=stm32mp1 AARCH32_SP=sp_min \
  DTB_FILE_NAME=stm32mp157a-dk1.dtb BL33=../u-boot/u-boot-nodtb.bin \
  BL33_CFG=../u-boot/u-boot.dtb STM32MP_SDMMC=1 fip all
\end{bashinput}

At the end of the build, the important output files generated are
located in \code{build/stm32mp1/release/}. We will find there:

\begin{itemize}

\item \code{tf-a-stm32mp157a-dk1.stm32}, which is TF-A BL2, serving as
  our first stage bootloader

\item \code{fip.bin}, which is the FIP image, which itself includes
  U-Boot. This image will serve as the second stage bootloader.

\end{itemize}

\section{Flashing the bootloaders}

The ROM monitor will look for the first stage bootloader in a
partition named \code{fsbl1}. If it cannot find a valid bootloader in
this partition, it will then try to load it from a partition named
\code{fslb2}. This first stage bootloader (in our case the TF-A BL2)
will load the second bootloader (U-Boot) from the Firmware Image
Package located in the partition named \code{fip}. At the same time,
BL2 will also load the BL32 monitor (SP-min) from the FIP. Finally,
U-Boot will store its environment in the fourth partition, which we'll
name \code{bootfs}.

So, as far as bootloaders are concerned, the SD card partitioning will
look like:

\begin{verbatim}
Number  Start   End      Size     File system  Name    Flags
 1      2048s   4095s    2048s                 fsbl1
 2      4096s   6143s    2048s                 fsbl2
 3      6144s   10239s   4096s                 fip
 4      10240s  131071s  120832s               bootfs
\end{verbatim}

On your workstation, plug in the SD card your instructor gave you. Type
the \code{sudo dmesg} command to see which device is used by your
workstation. In case the device is \code{/dev/mmcblk0}, you will see
something like

\begin{verbatim}
[46939.425299] mmc0: new high speed SDHC card at address 0007
[46939.427947] mmcblk0: mmc0:0007 SD16G 14.5 GiB
\end{verbatim}

The device file name may be different (such as \code{/dev/sdb}
if the card reader is connected to a USB bus (either internally
or using a USB card reader).

In the following instructions, we will assume that your SD card is
seen as \code{/dev/mmcblk0} by your PC workstation.

Type the \code{mount} command to check your currently mounted
partitions. If SD partitions are mounted, unmount them:

\bashcmd{$ sudo umount /dev/mmcblk0p*}

We will erase the existing partition table and partition contents
by simply zero-ing the first 128 MiB of the SD card:

\bashcmd{$ sudo dd if=/dev/zero of=/dev/mmcblk0 bs=1M count=128}

Now, let's use the \code{parted} command to create the partitions that
we are going to use:

\bashcmd{$ sudo parted /dev/mmcblk0}

The ROM monitor handles {\em GPT} partition tables, let's create one:

\begin{verbatim}
(parted) mklabel gpt
\end{verbatim}

Then, the 4 partitions are created with:
\begin{verbatim}
(parted) mkpart fsbl1 0% 4095s
(parted) mkpart fsbl2 4096s 6143s
(parted) mkpart fip 6144s 10239s
(parted) mkpart bootfs 10240s 131071s
\end{verbatim}

You can verify everything looks right with:

\begin{verbatim}
(parted) print
Model: SD SA08G (sd/mmc)
Disk /dev/mmcblk0: 7747MB
Sector size (logical/physical): 512B/512B
Partition Table: gpt
Disk Flags:

Number  Start   End     Size    File system  Name    Flags
 1      1049kB  2097kB  1049kB               fsbl1
 2      2097kB  3146kB  1049kB               fsbl2
 3      3146kB  5243kB  2097kB               fip
 4      5243kB  67.1MB  61.9MB               bootfs

(parted)
\end{verbatim}

Once done, quit:
\begin{verbatim}
(parted) quit
\end{verbatim}

{\em Note: \code{parted} is definitely not very user friendly compared
to other tools to manipulate partitions (such as \code{cfdisk}), but
that's the only tool which supports assigning names to GPT partitions.
In your projects, you could use \code{gparted}, which is a more
friendly graphical front-end on top of \code{parted}.}

Now, format the boot partition as an ext4 filesystem. This is where
U-Boot saves its environment:
\bashcmd{$ sudo mkfs.ext4 -L boot -O ^metadata_csum /dev/mmcblk0p4}

The \code{-O ^metadata_csum} option allows to create the filesystem
without enabling metadata checksums, which U-Boot doesn't seem to
support yet.

Now write the TF-A binary in both \code{fsbl} partitions:

\begin{bashinput}
$ sudo dd if=build/stm32mp1/release/tf-a-stm32mp157a-dk1.stm32 of=/dev/mmcblk0p1 bs=1M conv=fdatasync
$ sudo dd if=build/stm32mp1/release/tf-a-stm32mp157a-dk1.stm32 of=/dev/mmcblk0p2 bs=1M conv=fdatasync
\end{bashinput}

Then flash the {\em fip} partition with the Firmware Image Package
containing U-Boot, the BL32 monitor and their configuration (device tree):

\bashcmd{$ sudo dd if=build/stm32mp1/release/fip.bin of=/dev/mmcblk0p3 bs=1M conv=fdatasync}

\section{Testing the bootloaders}

Insert the SD card in the board slot. You can now power-up the board
by connecting the USB-C cable to the board, in CN6, \code{PWR_IN} and
to your PC at the other end. Check that it boots your new bootloaders.
You can verify this by checking the build dates:

\begin{verbatim}
NOTICE:  CPU: STM32MP157DAC Rev.Z
NOTICE:  Model: STMicroelectronics STM32MP157A-DK1 Discovery Board
NOTICE:  Board: MB1272 Var3.0 Rev.C-02
NOTICE:  BL2: v2.9(release):v2.9.0
NOTICE:  BL2: Built : 15:13:02, May 20 2024
NOTICE:  BL2: Booting BL32
NOTICE:  SP_MIN: v2.9(release):v2.9.0
NOTICE:  SP_MIN: Built : 15:12:58, May 20 2024


U-Boot 2024.04 (May 20 2024 - 15:10:25 +0200)

CPU: STM32MP157DAC Rev.Z
Model: STMicroelectronics STM32MP157A-DK1 Discovery Board
Board: stm32mp1 in trusted mode (st,stm32mp157a-dk1)
Board: MB1272 Var3.0 Rev.C-02
DRAM:  512 MiB
Clocks:
- MPU : 650 MHz
- MCU : 208.878 MHz
- AXI : 266.500 MHz
- PER : 24 MHz
- DDR : 533 MHz
Core:  306 devices, 35 uclasses, devicetree: board
NAND:  0 MiB
MMC:   STM32 SD/MMC: 0
Loading Environment from EXT4... ** File not found /uboot.env **

** Unable to read "/uboot.env" from mmc0:4 **
In:    serial
Out:   serial
Err:   serial
Previous ADC measurements was not the one expected, retry in 20ms
****************************************************
*        WARNING 500mA power supply detected       *
*     Current too low, use a 3A power supply!      *
****************************************************

Net:   eth0: ethernet@5800a000
Hit any key to stop autoboot:  0
STM32MP>
\end{verbatim}

In U-Boot, type the \code{help} command, and explore the few commands
available.

\subsection{Adding a new command to the U-Boot shell}

Check whether the \code{config} command is available. This command
allows to dump the configuration settings U-Boot was compiled from.

If it's not, go back to U-Boot's configuration and enable it.

Re-run the build of U-Boot, and then re-run the build of TF-A so that
a new version of the \code{fip.bin} with the updated U-Boot is
generated.

Update the \code{fip} partition on the SD card with the new
\code{fip.bin} image and test that the command is now available and
works as expected.

\subsection{Playing with the U-Boot environment}

Display the U-Boot environment using \code{printenv}.

Set a new U-Boot variable \code{foo} to a value of your choice, using
\code{setenv}, and verify it has been set. Reset the board, and check
if \code{foo} is still defined: it should not.

Now repeat this process, but before resetting the board, use
\code{saveenv}. After the reset, check the \code{foo} variable is
still defined.

Now reset the environment to its default settings using \code{env
  default -a}, and save these changes using \code{saveenv}.

\section{Setting up networking}

The next step is to configure U-boot and your workstation to let your
board download files, such as the kernel image and Device Tree Binary
(DTB), using the TFTP protocol through a network connection.

With a network cable, connect the Ethernet port of
your board to the one of your computer. If your computer already has a
wired connection to the network, your instructor will provide you with
a USB Ethernet adapter. A new network interface should appear on your
Linux system.

\subsection{Network configuration on the target}

Let's configure networking in U-Boot:

\begin{itemize}
  \item \code{ipaddr}: IP address of the board
  \item \code{serverip}: IP address of the PC host
\end{itemize}

\begin{ubootinput}
=> setenv ipaddr 192.168.0.100
=> setenv serverip 192.168.0.1
\end{ubootinput}

Of course, make sure that this address belongs to a separate network
segment from the one of the main company network.

To make these settings permanent, save the environment:

\begin{ubootinput}
=> saveenv
\end{ubootinput}

\subsection{Network configuration on the PC host}

To configure your network interface on the workstation side, we need
to know the name of the network interface connected to your board.

Find the name of this interface by typing:
\bashcmd{=> ip a}

The network interface name is likely to be
\code{enxxx}\footnote{Following the {\em Predictable Network Interface
Names} convention:
\url{https://www.freedesktop.org/wiki/Software/systemd/PredictableNetworkInterfaceNames/}}.
If you have a pluggable Ethernet device, it's easy to identify as it's
the one that shows up after pluging in the device.

Then, instead of configuring the host IP address from NetworkManager's
graphical interface, let's do it through its command line interface,
which is so much easier to use:

\bashcmd{$ nmcli con add type ethernet ifname en... ip4 192.168.0.1/24}

\section{Setting up the TFTP server}

Let's install a TFTP server on your development workstation:

\begin{verbatim}
sudo apt install tftpd-hpa
\end{verbatim}

You can then test the TFTP connection. First, put a small text file in
the directory exported through TFTP on your development
workstation. Then, from U-Boot, do:

\begin{ubootinput}
=> tftp %\zimageboardaddr% textfile.txt
\end{ubootinput}

The \code{tftp} command should have downloaded the
\code{textfile.txt} file from your development workstation into
the board's memory at location {\tt \zimageboardaddr}\footnote{
This location is part of the board DRAM. If you want
to check where this value comes from, you can check the SoC
datasheet at
\url{https://www.st.com/resource/en/reference_manual/dm00327659.pdf}.
It's a big document (more than 4,000 pages). In this document, look
for \code{Memory organization} and you will find the SoC memory map.
You will see that the address range for the memory controller
({\em DDRC})
starts at the address we are looking for.
You can also try with other values in the RAM address range.}.

You can verify that the download was successful by dumping the
contents of the memory:

\begin{ubootinput}
=> md %\zimageboardaddr%
\end{ubootinput}

We will see in the next labs how to use U-Boot to download, flash and
boot a kernel.

\input{../common/dk1-known-issues.tex}

\section{Rescue binaries}

If you have trouble generating binaries that work properly, or later
make a mistake that causes you to lose your bootloader binaries, you
will find working versions under \code{data/} in the current lab
directory.
