\section{Using Yocto Project - advanced usage}

\begin{frame}
  \frametitle{Advanced build usage and configuration}
  \begin{itemize}
    \item Variable operators and overrides.
    \item Select package variants.
    \item Manually add packages to the generated image.
    \item Run specific tasks with BitBake.
  \end{itemize}
\end{frame}

\begin{frame}
  \frametitle{A little reminder}
  \begin{itemize}
    \item A {\em Recipe} describes how to fetch, configure, compile and
      install a software component (application, library, \dots).
    \item These tasks can be run independently (if their dependencies
      are met).
    \item All the available packages in the project layer are not
      selected by default to be built and included in the images.
    \item Some packages may provide the same functionality, e.g.
      OpenSSH and Dropbear.
  \end{itemize}
\end{frame}

\subsection{Variables}

\begin{frame}
  \frametitle{Overview}
  \begin{itemize}
    \item The OpenEmbedded build system uses configuration {\em variables}
      to hold information.
    \item Variable {\em names} are in upper-case by convention, e.g.
      \code{CONF_VERSION}
    \item Variable {\em values} are strings
    \item To make configuration easier, it is possible to prepend,
      append or define these variables in a conditional way.
    \item Variables defined in \textbf{Configuration Files} have a \textbf{global} scope
      \begin{itemize}
      \item Files ending in \code{.conf}
      \end{itemize}
    \item Variables defined in \textbf{Recipes} have a \textbf{local} scope
      \begin{itemize}
      \item Files ending in \code{.bb}, \code{.bbappend} and \code{.bbclass}
      \end{itemize}
    \item Recipes can also access the global scope
  \end{itemize}
\end{frame}

\begin{frame}
  \frametitle{Operators}
  \begin{itemize}
    \item Various operators can be used to assign values to
    configuration variables:
      \begin{description}
        \item[=] expand the value when using the variable
        \item[:=] immediately expand the value
        \item[+=] append (with space)
        \item[=+] prepend (with space)
        \item[.=] append (without space)
        \item[=.] prepend (without space)
        \item[?=] assign if no other value was previously assigned
        \item[??=] same as previous, with a lower precedence
      \end{description}
  \end{itemize}
\end{frame}

\begin{frame}[fragile]
  \frametitle{Operators caveats}
  \begin{itemize}
    \item The operators apply their effect during parsing
    \item Example:
  \end{itemize}
  \begin{columns}
    \column{0.2\textwidth}
    \column{0.3\textwidth}
    \begin{block}{}
      \begin{minted}{text}
VAR ?= "a"
VAR += "b"
      \end{minted}
    \end{block}
    Result: \code{VAR = "a b"}
    \break
    \column{0.3\textwidth}
    \begin{block}{}
      \begin{minted}{text}
VAR += "b"
VAR ?= "a"
      \end{minted}
    \end{block}
    Result: \code{VAR = "b"}
    \break
    \column{0.2\textwidth}
  \end{columns}

  \begin{itemize}
    \item The parsing order of files is difficult to predict, no assumption
      should be made about it.
    \item To avoid the problem, avoid using \code{+=}, \code{=+}, \code{.=}
      and \code{=.} in \code{$BUILDDIR/conf/local.conf}. Always use
      overrides (see following slides).
  \end{itemize}
\end{frame}

\begin{frame}
  \frametitle{Overrides}
  \begin{itemize}
  \item Bitbake \textbf{overrides} allow appending, prepending or modifying
    a variable at expansion time, when the variable's value is read
  \item Overrides are written as \code{<VARIABLE>:<override> = "some_value"}
  \item A different syntax was used before \textbf{Honister} (3.4), with no
    retrocompatibility: \code{<VARIABLE>_<override> = "some_value"}
  \end{itemize}
\end{frame}

\begin{frame}
  \frametitle{Overrides to modify variable values}
  \begin{itemize}
    \item The \code{append} override adds {\bf at the end} of the variable
      (without space).
      \begin{itemize}
        \item \code{IMAGE_INSTALL:append = " dropbear"} adds
          \code{dropbear} to the packages installed on the image.
      \end{itemize}
    \item The \code{prepend} override adds {\bf at the beginning} of the
      variable (without space).
      \begin{itemize}
        \item \code{FILESEXTRAPATHS:prepend := "${THISDIR}/${PN}:"}
          %stopzone
          adds the folder to the set of paths where files are located
          (in a recipe).
      \end{itemize}
    \item The \code{remove} override removes all occurrences of a value
      within a variable.
      \begin{itemize}
        \item \code{IMAGE_INSTALL:remove = "i2c-tools"}
      \end{itemize}
  \end{itemize}
\end{frame}

\begin{frame}[fragile]
  \frametitle{Overrides for conditional assignment}
  \begin{itemize}
    \item Append the machine name to only define a configuration variable
      for a given machine.
    \item It tries to match with values from \yoctovar{OVERRIDES} which
      includes \yoctovar{MACHINE}, \yoctovar{SOC_FAMILY}, and more.
    \item If the override is in \yoctovar{OVERRIDES}, the assignment is
      applied, otherwise it is ignored.
  \end{itemize}
  \begin{block}{}
    \begin{minted}[fontsize=\footnotesize]{shell}
OVERRIDES="arm:armv7a:ti-soc:ti33x:beaglebone:poky"

KERNEL_DEVICETREE:beaglebone = "am335x-bone.dtb" # This is applied
KERNEL_DEVICETREE:dra7xx-evm = "dra7-evm.dtb"    # This is ignored
    \end{minted}
  \end{block}
\end{frame}

\begin{frame}[fragile]
  \frametitle{Overrides for conditional assignment: precedence}
  \begin{itemize}
    \item The most specific assignment takes precedence.
    \item Example:
      \begin{minted}[fontsize=\footnotesize]{console}
IMAGE_INSTALL:beaglebone = "busybox mtd-utils i2c-tools"
IMAGE_INSTALL = "busybox mtd-utils"
      \end{minted}
    \item If the machine is \code{beaglebone}:
      \begin{itemize}
        \item \code{IMAGE_INSTALL = "busybox mtd-utils i2c-tools"}
      \end{itemize}
    \item Otherwise:
      \begin{itemize}
        \item \code{IMAGE_INSTALL = "busybox mtd-utils"}
      \end{itemize}
  \end{itemize}
\end{frame}

\begin{frame}
  \frametitle{Combining overrides}
  \begin{itemize}
    \item The previous methods can be combined.
    \item If we define:
      \begin{itemize}
        \item \code{IMAGE_INSTALL = "busybox mtd-utils"}
        \item \code{IMAGE_INSTALL:append = " dropbear"}
        \item \code{IMAGE_INSTALL:append:beaglebone = " i2c-tools"}
      \end{itemize}
    \item The resulting configuration variable will be:
      \begin{itemize}
        \item \code{IMAGE_INSTALL = "busybox mtd-utils dropbear
          i2c-tools"} if the machine being built is
          \code{beaglebone}.
        \item \code{IMAGE_INSTALL = "busybox mtd-utils dropbear"}
          otherwise.
      \end{itemize}
  \end{itemize}
\end{frame}

\begin{frame}
  \frametitle{Order of variable assignment}
  \begin{columns}
    \column{0.4\textwidth}
      \includegraphics[width=\textwidth]{slides/yocto-advanced/yocto-operators-order.pdf}
    \column{0.6\textwidth}
      \begin{enumerate}
        \item All the operators are applied, in parsing order
        \item \code{:append} overrides are applied
        \item \code{:prepend} overrides are applied
        \item \code{:remove} overrides are applied
      \end{enumerate}
  \end{columns}
\end{frame}

\begin{frame}[fragile]
  \frametitle{\code{bitbake-getvar}}
  \begin{itemize}
    \item \code{bitbake-getvar} can be used to understand and debug
      how variables are assigned
    \item \code{bitbake-getvar <VARIABLE>}
    \item Lists each configuration file touching the variable, the
      pre-expansion value and the final value
  \end{itemize}
  \begin{block}{}
    \begin{minted}[fontsize=\tiny]{console}
$ bitbake-getvar DEPLOY_DIR
NOTE: Starting bitbake server...
#
# $DEPLOY_DIR [2 operations]
#   set? /home/user/yocto-labs/poky/meta/conf/bitbake.conf:440
#     "${TMPDIR}/deploy"
#   set /home/user/yocto-labs/poky/meta/conf/documentation.conf:137
#     [doc] "Points to the general area that the OpenEmbedded build system uses to place images, [...]"
# pre-expansion value:
#   "${TMPDIR}/deploy"
DEPLOY_DIR="/home/user/yocto-labs/build/tmp/deploy"
$
    \end{minted}
  \end{block}
\end{frame}

\subsection{Package variants}

\begin{frame}
  \frametitle{Introduction to package variants}
  \begin{itemize}
    \item Some packages have the same purpose, and only one can be
      used at a time.
    \item The build system uses {\bf virtual packages} to reflect
      this. A virtual package describes functionalities and several
      packages may provide it.
    \item Only one of the packages that provide the functionality will
    be compiled and integrated into the resulting image.
  \end{itemize}
\end{frame}

\begin{frame}
  \frametitle{Variant examples}
  \begin{itemize}
    \item The virtual packages are often in the form
      \code{virtual/<name>}
    \item Example of available virtual packages with some of their
      variants:
      \begin{itemize}
        \item \code{virtual/bootloader}: u-boot,
          u-boot-ti-staging\dots
        \item \code{virtual/kernel}: linux-yocto, linux-yocto-tiny,
          linux-yocto-rt, linux-ti-staging\dots
        \item \code{virtual/libc}: glibc, musl, newlib
        \item \code{virtual/xserver}: xserver-xorg
      \end{itemize}
  \end{itemize}
\end{frame}

\begin{frame}
  \frametitle{Package selection}
  \begin{itemize}
    \item Variants are selected thanks to the
      \yoctovar{PREFERRED_PROVIDER} configuration variable.
    \item The package names {\bf have to} suffix this variable.
    \item Examples:
    \begin{itemize}
      \item \code{PREFERRED_PROVIDER_virtual/kernel ?=
        "linux-ti-staging"}
      \item \code{PREFERRED_PROVIDER_virtual/libgl = "mesa"}
    \end{itemize}
  \end{itemize}
\end{frame}

\begin{frame}[fragile]
  \frametitle{Version selection}
  \begin{itemize}
    \item By default, Bitbake will try to build the provider with the
      highest version number, from the highest priority layer, unless the recipe defines
      \code{DEFAULT_PREFERENCE = "-1"}
    \item When multiple package versions are available, it is also
      possible to explicitly pick a given version with
      \yoctovar{PREFERRED_VERSION}.
    \item The package names {\bf have to} suffix this variable.
    \item {\bf \%} can be used as a wildcard.
    \item Example:
    \begin{itemize}
      \item \code{PREFERRED_VERSION_nginx = "1.20.1"}
      \item \usebeamercolor[fg]{code} \path{PREFERRED_VERSION_linux-yocto = "5.14%"}
    \end{itemize}
  \end{itemize}
\end{frame}

\subsection{Selection of packages to install}

\begin{frame}
  \frametitle{Selection of packages to install}
  \begin{itemize}
    \item The set of packages installed into the image is defined by
      the target you choose (e.g. \code{core-image-minimal}).
    \item It is possible to have a custom set by defining our own
      target, and we will see this later.
    \item When developing or debugging, adding packages can be useful,
      without modifying the recipes.
    \item Packages are controlled by the \yoctovar{IMAGE_INSTALL}
      configuration variable.
  \end{itemize}
\end{frame}

\subsection{The power of BitBake}

\begin{frame}
  \frametitle{Common BitBake options}
  \begin{itemize}
    \item BitBake can be used to run a full build for a given target
      with \code{bitbake [target]}
    \begin{itemize}
      \item \code{target} is a recipe name, possibly with modifiers,
        e.g. \code{-native}
      \item \code{bitbake ncurses}
      \item \code{bitbake ncurses-native}
    \end{itemize}
    \item But it can be more precise, with additional options:
    \begin{description}
      \item[\code{-c <task>}] execute the given task
      \item[\code{-s}] list all available recipes and their
        versions
      \item[\code{-f}] force the given task to be run by removing its
        stamp file
      \item[\code{world}] keyword for all recipes
    \end{description}
  \end{itemize}
\end{frame}

\begin{frame}
  \frametitle{BitBake examples}
  \begin{itemize}
    \item \code{bitbake -c listtasks virtual/kernel}
    \begin{itemize}
      \item Gives a list of the available tasks for the recipe
        providing the package \code{virtual/kernel}. Tasks are
        prefixed with \code{do_}.
    \end{itemize}
    \item \code{bitbake -c menuconfig virtual/kernel}
    \begin{itemize}
      \item Execute the task \code{menuconfig} on the recipe providing
        the \code{virtual/kernel} package.
    \end{itemize}
    \item \code{bitbake -f dropbear}
    \begin{itemize}
      \item Force the \code{dropbear} recipe to run all tasks.
    \end{itemize}
    \item \code{bitbake --runall=fetch core-image-minimal}
    \begin{itemize}
      \item Download all recipe sources and their dependencies.
    \end{itemize}
    \item For a full description: \code{bitbake --help}
  \end{itemize}
\end{frame}

\begin{frame}[fragile]
  \frametitle{shared state cache}
  \begin{itemize}
    \item BitBake stores the output of each task in a directory, the
      shared state cache.
    \item This cache is used to speed up compilation.
    \item Its location is defined by the \yoctovar{SSTATE_DIR} variable and
      defaults to \code{build/sstate-cache}.
    \item Over time, as you compile more recipes, it can grow quite
      big. It is possible to clean old data with:
      \begin{block}{}
      \begin{minted}[fontsize=\footnotesize]{console}
$ find sstate-cache/ -type f -atime +30 -delete
      \end{minted}
      \end{block}
      This removes all files that have last been accessed more than 30 days ago
      (for example).
  \end{itemize}
\end{frame}
